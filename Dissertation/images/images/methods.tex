% Inkscape figure
\begin{figure}
    {\centering
      %\def\svgwidth{5cm} % используем для изменения размера, если надо
      %\includesvg{figs/drawing}
      \subbottom[Метод Казагранде]{%
      \input{Dissertation/images/oedometerCazagrande-1.pdf_tex} }
    %\hfill
      \subbottom[Метод Беккера]{%
      \input{Dissertation/images/oedometerBecker-1.pdf_tex}}
      }
      \caption{Образец \texttt{GJ6835}}
      \label{img:6835}
    \end{figure}
    
    \begin{figure}
        {\centering
          %\def\svgwidth{5cm} % используем для изменения размера, если надо
          %\includesvg{figs/drawing}
          \subbottom[Метод Казагранде]{%
          \input{Dissertation/images/oedometerCazagrande-2.pdf_tex} }
        %\hfill
          \subbottom[Метод Беккера]{%
          \input{Dissertation/images/oedometerBecker-2.pdf_tex}}
          }
          \caption{Образец \texttt{GJ6874}}
          \label{img:6874}
    \end{figure}
    
    \begin{figure}
        {\centering
            %\def\svgwidth{5cm} % используем для изменения размера, если надо
            %\includesvg{figs/drawing}
            \subbottom[Метод Казагранде]{%
            \input{Dissertation/images/oedometerCazagrande-3.pdf_tex} }
        %\hfill
            \subbottom[Метод Беккера]{%
            \input{Dissertation/images/oedometerBecker-3.pdf_tex}}
            }
            \caption{Образец \texttt{GJ6890}}
            \label{img:6890}
    \end{figure}
    
    
    \begin{figure}
        {\centering
            %\def\svgwidth{5cm} % используем для изменения размера, если надо
            %\includesvg{figs/drawing}
            \subbottom[Метод Казагранде]{%
            \input{Dissertation/images/oedometerCazagrande-4.pdf_tex} }
        %\hfill
            \subbottom[Метод Беккера]{%
            \input{Dissertation/images/oedometerBecker-4.pdf_tex}}
            }
            \caption{Образец \texttt{GJ6864}}
            \label{img:6864}
    \end{figure}
    
    \begin{figure}
        {\centering
            %\def\svgwidth{5cm} % используем для изменения размера, если надо
            %\includesvg{figs/drawing}
            \subbottom[Метод Казагранде]{%
            \input{Dissertation/images/oedometerCazagrande-5.pdf_tex} }
        %\hfill
            \subbottom[Метод Беккера]{%
            \input{Dissertation/images/oedometerBecker-5.pdf_tex}}
            }
            \caption{Образец \texttt{GJ6838}}
            \label{img:6838}
    \end{figure}
    
    \begin{figure}
        {\centering
            %\def\svgwidth{5cm} % используем для изменения размера, если надо
            %\includesvg{figs/drawing}
            \subbottom[Метод Казагранде]{%
            \input{Dissertation/images/oedometerCazagrande-6.pdf_tex} }
        %\hfill
            \subbottom[Метод Беккера]{%
            \input{Dissertation/images/oedometerBecker-6.pdf_tex}}
            }
            \caption{Образец \texttt{GJ6898}}
            \label{img:6898}
    \end{figure}
    
    \begin{figure}
        {\centering
            %\def\svgwidth{5cm} % используем для изменения размера, если надо
            %\includesvg{figs/drawing}
            \subbottom[Метод Казагранде]{%
            \input{Dissertation/images/oedometerCazagrande-7.pdf_tex} }
        %\hfill
            \subbottom[Метод Беккера]{%
            \input{Dissertation/images/oedometerBecker-7.pdf_tex}}
            }
            \caption{Образец \texttt{GJ6888}}
            \label{img:6888}
    \end{figure}
    
    \begin{figure}
        {\centering
            %\def\svgwidth{5cm} % используем для изменения размера, если надо
            %\includesvg{figs/drawing}
            \subbottom[Метод Казагранде]{%
            \input{Dissertation/images/oedometerCazagrande-8.pdf_tex} }
        %\hfill
            \subbottom[Метод Беккера]{%
            \input{Dissertation/images/oedometerBecker-8.pdf_tex}}
            }
            \caption{Образец \texttt{GJ68A0}}
            \label{img:68A0}
    \end{figure}
    
    \begin{figure}
        {\centering
            %\def\svgwidth{5cm} % используем для изменения размера, если надо
            %\includesvg{figs/drawing}
            \subbottom[Метод Казагранде]{%
            \input{Dissertation/images/oedometerCazagrande-9.pdf_tex} }
        %\hfill
            \subbottom[Метод Беккера]{%
            \input{Dissertation/images/oedometerBecker-9.pdf_tex}}
            }
            \caption{Образец \texttt{GJ6840}}
            \label{img:6840}
    \end{figure}
    
    \begin{figure}
        {\centering
            %\def\svgwidth{5cm} % используем для изменения размера, если надо
            %\includesvg{figs/drawing}
            \subbottom[Метод Казагранде]{%
            \input{Dissertation/images/oedometerCazagrande-10.pdf_tex} }
        \hfill
            \subbottom[Метод Беккера]{%
            \input{Dissertation/images/oedometerBecker-10.pdf_tex}}
            }
            \caption{Образец \texttt{GJ6895}}
            \label{img:6895}
    \end{figure}
    
    \begin{figure}
        {\centering
            %\def\svgwidth{5cm} % используем для изменения размера, если надо
            %\includesvg{figs/drawing}
            \subbottom[Метод Казагранде]{%
            \input{Dissertation/images/oedometerCazagrande-11.pdf_tex} }
        %\hfill
            \subbottom[Метод Беккера]{%
            \input{Dissertation/images/oedometerBecker-11.pdf_tex}}
            }
            \caption{Образец \texttt{GJ6885}}
            \label{img:6885}
    \end{figure}
    
    \begin{figure}
        {\centering
            %\def\svgwidth{5cm} % используем для изменения размера, если надо
            %\includesvg{figs/drawing}
            \subbottom[Метод Казагранде]{%
            \input{Dissertation/images/oedometerCazagrande-12.pdf_tex} }
        %\hfill
            \subbottom[Метод Беккера]{%
            \input{Dissertation/images/oedometerBecker-12.pdf_tex}}
            }
            \caption{Образец \texttt{GJ68B3}}
            \label{img:68B3}
    \end{figure}