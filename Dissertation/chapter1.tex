\chapter{Современные представления о методах определения коэффициента переуплотнения}



Карл Терцаги (1883—1963)— австрийский и американский геолог и инженер-строитель, один из основоположников механики грунтов.

Из ранних исследований профессора Карла Терцаги по механике уплотнения мелкозернистых грунтов можно сделать вывод, что зависимость между соотношением пустот и давлением для первичного или первичного разветвления кривой сжатия может быть выражена логарифмически. Обширные испытания показали, что такое логарифмическое отношение справедливо по крайней мере до 2 МПа, то есть для всего диапазона нагрузок, которые используются в гражданском строительстве. Любые существенные отклонения между кривой сжатия ненарушенного глинистого образца, по-видимому, вызваны вариациями нагрузки, которую испытывает грунт в геологической истории, и при его удалении из грунта. Это можно определить, исходя из формы кривой разгрузки и повторного сжатия, полученной при нагружении образца в условиях, значительно превышающих напряжение, которое испытывал образец грунта, находясь в массиве. Нагрузка снижается до нуля, потом снова постепенно увеличивается до еще большей нагрузки. (<<The determination>> А. Казагранде 1936). \cite{boone_critical_2010}
 
 Опыты профессора Терцаги показали, что для водонасыщенных маловодопроницаемых глинистых грунтов при каждом увеличении нагрузки соответствует определенное изменение влажности. Зависимость влажности от нагрузки можно изобразить в виде кривой, которая называется компрессионной. Между влажностью и коэффициентом пористости у полностью водонасыщенных грунтов существует связь, поэтому компрессионную кривую можно построить в графике зависимости давления от коэффициента пористости. Если начертить компрессионную кривую в полулогарифмическом масштабе, тогда изменения коэффициента пористости будет линейно зависеть от логарифма изменения приложенной нагрузки. Уравнение компрессионной кривой в этом случае будет выглядеть следующим образом:
 $$e_i=e_0-C_c\ln(\frac{p_i}{p_0})$$
 
 
 Коэффициент компрессии $C_c$ - есть тангенс угла наклона полулогарифмической кривой к оси давлений. Он численно равен разности коэффициента пористости при давлении на данной ступени нагрузки(0,272 МПа) и при начальном давлении (0,1 МПа). Этот коэффициент дает характеристику сжимаемости грунтов в большом диапазоне давлений. 
 
 Исходя из этих умозаключений был сформулирован закон, который иммет особо важное значение в механике грунтов и кладется в основу установления ряда ее фундаментальных положений: принципа линейной деформируемости, принципа гидроемкости, дифференциального уравнения консолидации и прочих. Этот закон называется "закон уплотнения грунтов" и формулируется так: "Бесконечно малое изменение относительного объема пор грунта прямо пропорционально бесконечно малому изменению давления."("Механика грунтов" Н.А. Цытович 1983)
 Cam Clay


 
 Артур Казагранде (1902-1981) - американский инженер-строитель австрийского происхождения, внесший значительный вклад в области инженерной геологии и геотехники. Известный своими изобретательными разработками приборов для испытания грунта и фундаментальными исследованиями по просачиванию и разжижению грунта, он также известен разработкой программы преподавания механики грунта в Гарвардском университете в начале 1930-х годов, которая с тех пор была смоделирована во многих университетах по всему миру. 
 
 Метод Казагранде был предложен Артуром Казагранде в 1936 году. Он стал одним из основных и самых распространенных методов определения давления предварительного уплотнения, который имеет широкое распространение и в настоящее время. 
 
  Это графический метод, который основан на определении точки максимальной кривизны на графике компрессионной кривой, построенной в полулогарифмическом масштабе – зависимость коэффициента пористости или деформации от логарифма вертикального (эффективного) напряжения. Через эту точку проводится касательная к участку кривой, расположенному за точкой перегиба и горизонтальная линия. Через угол, который образуют эти две прямые проводится биссектриса. Далее находится точка пересечения биссектрисы угла и «продолжения» прямолинейного участка компрессионной кривой. Проекция точки пересечения на ось напряжений, есть величина давления предварительного уплотнения. Хотя метод хорошо зарекомендовал себя для анализа глинистых грунтов, однако, при построении плавной компрессионной кривой значительно усложняются условия решения задачи, потому как отсутствует участок кривой, где можно выделить видимое уплотнение грунта. Так же получаемые значения давления зависят от выбранного масштаба графика. 
  
  Несколько иной способ обработки данных был предложен Д. Беккером. Согласно этому методу значение $\sigma_p$ находится по графику зависимости увеличения работы на единицу объема (произведения давления на деформацию) от приращения вертикального давления. Сама процедура определения заключается в следующем: Вначале вычисляется изменение работы на единицу объема для каждого приращения деформации. Далее строится график зависимости работы А от вертикального давления $\sigma$. Величина давления (суммарная работа)  будет определена давлением в конце приращения деформации.Напряжение в точке пересечения двух прямолинейных участков, которые получаются после данной зависимости, и будет соответствовать давлению предварительного уплотнения.
  
   Эта методика была применена к результатам испытаний в одометре, проведенных на образцах глинистых грунтов природного сложения и на образцах, консолидированных анизотропно из вязкого слоя с известным эффективным напряжением. Соотношение работы на единицу объема-эффективное напряжение, может быть аппроксимировано или установлено с помощью линейных соотношений. Показано, что пересечения этих прямых дают точные значения напряжений и предела текучести (предварительного уплотнения). Предел текучести определяется как пересечение исходной линии и линейной зависимости, наблюдаемой при более высоких напряжениях. На текущее эффективное напряжение указывает значительное расхождение данных с исходной установленной линией. Эти соотношения применимы как к условно (горизонтально) обрезанным образцам, так и к вертикально обрезанным образцам одометра. Выдвинута гипотеза, что действующие эффективные напряжения (как в вертикальном, так и в горизонтальном направлениях) в природной глине могут быть определены работой на единицу объема интерпретации компрессионных испытаний, проведенных на горизонтально и вертикально обрезанных образцах. ("критерии для определения природных и бытовых напряжений в глинах" D.Becker at all, 1987)
  
  Чтобы обеспечить надежность и достоверность полученных значений, а так же снизить риск возможности ошибки в ходе обработки результатов, рекомендуется использовать оба метода обработки параллельно.
  