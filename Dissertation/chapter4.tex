\chapter{Результаты исследований методов определения напряжения предуплотнения и их обсуждение}

Чтобы оценить результаты испытаний, проведенных в ходе исследования, было использовано два подхода.

Первый подход "--- статистический. Он заключается в том, чтобы провести определение напряжения предуплотнения разными специалистами и сравнить полученные результаты. Этот подход охарактеризует субъективный фактор при использовании методов и его влияние, о которых говорилось выше.

Второй подход заключается в приближении компрессионной кривой аналитической зависимостью. В качестве аналитического приближения выбран сплайн, представленный кривой Безье третьего порядка. Фиттинг (fitting) кривой осуществлялся вручную. Для полученной аналитической кривой вычислялись необходимые показатели, которые используются для определения напряжения предуплотнения методами Казагранде и Беккера. 

В процессе компрессионых испытаний грунтов были обнаружены некоторые сложности, связанные с неполадками приборов, которые в свою очередь повлияли на способы обработки данных. В результате этого факта не везде в полученных данных прослеживается логическая цепочка. Для опытов, которые остановились от независящих от нас причин, были разработаны меры по устранению неопределенностей с помощью фиксирования физических характеристик в начале и конце опыта, а также исходя из предположений о прямолинейности участков компрессионной кривой в логарифмическом масштабе. Однако, это не могло не отразится на погрешности в точности данных испытаний конкретных образцов. 

Вследствие чего, была проведена работа с пересчетом по формуле \ref{eq:ke}.
 %Результаты пересчета приведены ниже:


Исследования, проведенные на образцах нарушенной структуры, дают оценку точности данных методов. Исходя из полученных результатов компрессионных испытаний можно сделать вывод, что их обработка этими методами требует высокой квалификации от исследователя. Кроме того методы рекомендовано использовать совместно, чтобы снизить вероятность ошибки в расчетах. Несмотря, на свою многолетнюю историю, методы используют при исследовании оснований промышленно-гражданских сооружений и по сей день. 

Судя по результатам, которые были получены в ходе работы, можно сказать,что в обработке данных по методу Беккера требуется увеличение сгущения ступеней в зоне предполагаемого напряжения предуплотнения, потому как точность проведения касательных и точки их пересечения главным образом влияет на полученные значения.

Метод, предложенный Артуром Казагранде хорошо зарекомендовал себя для глинистых грунтов, однако, графические методы несколько неудобны для инженерного использования, в чем пришлось убедиться и на личном опыте.

В логарифмических координатах идеализированная компрессионная кривая в довольно широком интервале вертикальных напряжений представляет собой линейную зависимость в случае непереуплотненного грунта, и билинейную "--- в случае переуплотненного, о чем свидетельствуют результаты обработки компрессионных испытаний (приложение \ref{app:ct}.), которые были проведены в ходе данной исследовательской работы. 
%%\documentclass[border=10pt]{standalone}
%\documentclass[a4paper]{article}
%\usepackage{icomma}
%\usepackage{pgfplots}
%\usepackage{pgfplotstable}
%\pgfplotsset{compat=newest}

%\usepackage{filecontents}
%\usepackage{tikz}
%\pgfplotsset{width=0.4,compat=1.8}
%\usepgfplotslibrary{ternary}
%\usetikzlibrary{plotmarks}

%\newcommand{\q}[2]{#1 \textit{#2}}

%\begin{document}

\pgfplotsset{
    discard if not/.style 2 args={
        x filter/.code={
            \edef\tempa{\thisrow{#1}}
            \edef\tempb{#2}
            \ifx\tempa\tempb
            \else
                \def\pgfmathresult{inf}
            \fi
        }
    }
}

\makeatletter
\pgfplotstableset{
    discard if not/.style 2 args={
        row predicate/.code={
            \def\pgfplotstable@loc@TMPd{\pgfplotstablegetelem{##1}{#1}\of}
            \expandafter\pgfplotstable@loc@TMPd\pgfplotstablename
            \edef\tempa{\pgfplotsretval}
            \edef\tempb{#2}
            \ifx\tempa\tempb
            \else
                \pgfplotstableuserowfalse
            \fi
        }
    }
}
\makeatother

\pgfplotsset{
	%samples=15,
	width=0.7\linewidth,
	xlabel={Вертикальное напряжение $\sigma_1$, кПа},
	ylabel={Коэффициент пористости $e$},
	%extra y ticks={45},
	legend pos=north west,
	y tick label style={
		/pgf/number format/.cd,
			fixed,
			fixed zerofill,
			precision=3,
		/tikz/.cd
	},
	x tick label style={
		/pgf/number format/.cd,
			fixed,
			fixed zerofill
		/tikz/.cd
	}
}
\begin{tikzpicture}
	\begin{semilogxaxis}
	\addplot[mark=*, red] table [x=Sigma, y=Void, col sep=semicolon] {Dissertation/charts/GJ6835s.csv};
	\end{semilogxaxis}
\end{tikzpicture}

\begin{tikzpicture}
	\begin{semilogxaxis}
	\addplot[mark=*, red] table [x=Sigma, y=Void, col sep=semicolon] {Dissertation/charts/GJ6874s.csv};
	\end{semilogxaxis}
\end{tikzpicture}

\begin{tikzpicture}
	\begin{semilogxaxis}
	\addplot[mark=*, red] table [x=Sigma, y=Void, col sep=semicolon] {Dissertation/charts/GJ6890s.csv};
	\end{semilogxaxis}
\end{tikzpicture}


\begin{tikzpicture}
	\begin{semilogxaxis}
	\addplot[mark=*, red] table [x=Sigma, y=Void, col sep=semicolon] {Dissertation/charts/GJ6864s.csv};
	\end{semilogxaxis}
\end{tikzpicture}



\begin{tikzpicture}
	\begin{semilogxaxis}
	\addplot[mark=*, red] table [x=Sigma, y=Void, col sep=semicolon] {Dissertation/charts/GJ6838s.csv};
	\end{semilogxaxis}
\end{tikzpicture}

\begin{tikzpicture}
	\begin{semilogxaxis}
	\addplot[mark=*, red] table [x=Sigma, y=Void, col sep=semicolon] {Dissertation/charts/GJ6898s.csv};
	\end{semilogxaxis}
\end{tikzpicture}

\begin{tikzpicture}
	\begin{semilogxaxis}
	\addplot[mark=*, red] table [x=Sigma, y=Void, col sep=semicolon] {Dissertation/charts/GJ6888s.csv};
	\end{semilogxaxis}
\end{tikzpicture}

\begin{tikzpicture}
	\begin{semilogxaxis}
	\addplot[mark=*, red] table [x=Sigma, y=Void, col sep=semicolon] {Dissertation/charts/GJ68A0s.csv};
	\end{semilogxaxis}
\end{tikzpicture}


\begin{tikzpicture}
	\begin{semilogxaxis}
	\addplot[mark=*, red] table [x=Sigma, y=Void, col sep=semicolon] {Dissertation/charts/GJ6840s.csv};
	\end{semilogxaxis}
\end{tikzpicture}

\begin{tikzpicture}
	\begin{semilogxaxis}
	\addplot[mark=*, red] table [x=Sigma, y=Void, col sep=semicolon] {Dissertation/charts/GJ6895s.csv};
	\end{semilogxaxis}§
\end{tikzpicture}

\begin{tikzpicture}
	\begin{semilogxaxis}
	\addplot[mark=*, red] table [x=Sigma, y=Void, col sep=semicolon] {Dissertation/charts/GJ6885s.csv};
	\end{semilogxaxis}
\end{tikzpicture}

\begin{tikzpicture}
	\begin{semilogxaxis}
	\addplot[mark=*, red] table [x=Sigma, y=Void, col sep=semicolon] {Dissertation/charts/GJ68B3s.csv};
	\end{semilogxaxis}
\end{tikzpicture}
%\end{document}
%%\documentclass[border=10pt]{standalone}
%\documentclass[a4paper]{article}
%\usepackage{icomma}
%\usepackage{pgfplots}
%\usepackage{pgfplotstable}
%\pgfplotsset{compat=newest}

%\usepackage{filecontents}
%\usepackage{tikz}
%\pgfplotsset{width=0.4,compat=1.8}
%\usepgfplotslibrary{ternary}
%\usetikzlibrary{plotmarks}

%\newcommand{\q}[2]{#1 \textit{#2}}

%\begin{document}

\pgfplotsset{
    discard if not/.style 2 args={
        x filter/.code={
            \edef\tempa{\thisrow{#1}}
            \edef\tempb{#2}
            \ifx\tempa\tempb
            \else
                \def\pgfmathresult{inf}
            \fi
        }
    }
}

\makeatletter
\pgfplotstableset{
    discard if not/.style 2 args={
        row predicate/.code={
            \def\pgfplotstable@loc@TMPd{\pgfplotstablegetelem{##1}{#1}\of}
            \expandafter\pgfplotstable@loc@TMPd\pgfplotstablename
            \edef\tempa{\pgfplotsretval}
            \edef\tempb{#2}
            \ifx\tempa\tempb
            \else
                \pgfplotstableuserowfalse
            \fi
        }
    }
}
\makeatother

\pgfplotsset{
	%samples=15,
	width=0.7\linewidth,
	xlabel={Вертикальное напряжение $\sigma_1$, кПа},
	ylabel={Энергия $W$, кДж/м$^3$},
	%extra y ticks={45},
	legend pos=north west,
	y tick label style={
		/pgf/number format/.cd,
			fixed,
			fixed zerofill,
			precision=3,
		/tikz/.cd
	},
	x tick label style={
		/pgf/number format/.cd,
			fixed,
			fixed zerofill
		/tikz/.cd
	}
}
\begin{tikzpicture}
	\begin{axis}
	\addplot[mark=*, red] table [x=Sigma, y=Energy, col sep=semicolon] {Dissertation/charts/GJ6835s.csv};
	\end{axis}
\end{tikzpicture}

\begin{tikzpicture}
	\begin{axis}
	\addplot[mark=*, red] table [x=Sigma, y=Energy, col sep=semicolon] {Dissertation/charts/GJ6874s.csv};
	\end{axis}
\end{tikzpicture}

\begin{tikzpicture}
	\begin{axis}
	\addplot[mark=*, red] table [x=Sigma, y=Energy, col sep=semicolon] {Dissertation/charts/GJ6890s.csv};
	\end{axis}
\end{tikzpicture}


\begin{tikzpicture}
	\begin{axis}
	\addplot[mark=*, red] table [x=Sigma, y=Energy, col sep=semicolon] {Dissertation/charts/GJ6864s.csv};
	\end{axis}
\end{tikzpicture}



\begin{tikzpicture}
	\begin{axis}
	\addplot[mark=*, red] table [x=Sigma, y=Energy, col sep=semicolon] {Dissertation/charts/GJ6838s.csv};
	\end{axis}
\end{tikzpicture}

\begin{tikzpicture}
	\begin{axis}
	\addplot[mark=*, red] table [x=Sigma, y=Energy, col sep=semicolon] {Dissertation/charts/GJ6898s.csv};
	\end{axis}
\end{tikzpicture}

\begin{tikzpicture}
	\begin{axis}
	\addplot[mark=*, red] table [x=Sigma, y=Energy, col sep=semicolon] {Dissertation/charts/GJ6888s.csv};
	\end{axis}
\end{tikzpicture}

\begin{tikzpicture}
	\begin{axis}
	\addplot[mark=*, red] table [x=Sigma, y=Energy, col sep=semicolon] {Dissertation/charts/GJ68A0s.csv};
	\end{axis}
\end{tikzpicture}


\begin{tikzpicture}
	\begin{axis}
	\addplot[mark=*, red] table [x=Sigma, y=Energy, col sep=semicolon] {Dissertation/charts/GJ6840s.csv};
	\end{axis}
\end{tikzpicture}

\begin{tikzpicture}
	\begin{axis}
	\addplot[mark=*, red] table [x=Sigma, y=Energy, col sep=semicolon] {Dissertation/charts/GJ6895s.csv};
	\end{axis}§
\end{tikzpicture}

\begin{tikzpicture}
	\begin{axis}
	\addplot[mark=*, red] table [x=Sigma, y=Energy, col sep=semicolon] {Dissertation/charts/GJ6885s.csv};
	\end{axis}
\end{tikzpicture}

\begin{tikzpicture}
	\begin{axis}
	\addplot[mark=*, red] table [x=Sigma, y=Energy, col sep=semicolon] {Dissertation/charts/GJ68B3s.csv};
	\end{axis}
\end{tikzpicture}
%\end{document}

Основные принципы и методы таких измерений регламентированы шестью международными стандартами от ГОСТ Р ИСО 5725-1-2002 до ГОСТ Р ИСО 5725-6-2002 \cite{gost5725}, где основными показателями точности измерений являются <<правильность>> и <<прецизионность>>. Термин <<правильность>> характеризует близость измеренного среднего арифметического к принятому опорному значению, а <<прецизионность>> "--- степень близости измеренных значений друг к другу. Показателем правильности обычно является поправка, вычисляемая как разность между опорным и средним арифметическим значениями, прецизионность оценивается СКО (среднее квадратичное отклонение) \cite{pravandprec}.

Из полученных значений напряжения предуплотнения $\sigma_p$ (preconsolidation pressure), можно провести расчет напряжения переуплотнения $POP$ (preoverburden pressure) и коэффициента переуплотнения $OCR$ (overconsolidation ratio), а также показатели <<правильности>> и <<прецизионности>>. Получившиеся значения параметров приведены в таблице \ref{tab:OCR}.



Исходя из получившихся статистических значений, можно сделать вывод, что об определении параметров с высокой точностью говорить не приходится. Погрешность, которая появляется вследствие субъективного фактора определения существенно влияет на точность определения параметров переуплотнения грунта. Например, значение напряжения, определенные разными специалистами на одном образце, могут существенно различаться. В некоторых случаях отличия достигают до 250-300 кПа. Так у одного и того же образца давление переуплотнения $POP$ при бытовом давлении $\sigma_0 = 500$ кПа может варьироваться от 100 до 300 кПа, а коэффициент переуплотнения $OCR$ "--- от 1.2 до 1.6 .

Однако, методы хорошо себя зарекомендовали применительно к обработке компрессионных кривых, которые были получены после испытаний на образцах нарушенной структуры (пасты). В данных условиях было установлено, что значения, получившиеся в результате обработки методом Казагранде уступают в точности результатам обработки методом Беккера. Интуитивно можно понять, что это в первую очередь зависит от количества графических построений, которых в методе Беккера значительно меньше. Как говорилось ранее, параметры $OCR$ и $POP$ расчитываются по формулам $POP = \sigma_p - \sigma_0$ и $OCR = \frac{\sigma_p}{\sigma_0}$, соответственно. 
    