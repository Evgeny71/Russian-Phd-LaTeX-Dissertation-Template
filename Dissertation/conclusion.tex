\chapter*{Заключение}                       % Заголовок
\addcontentsline{toc}{chapter}{Заключение}  % Добавляем его в оглавление

%% Согласно ГОСТ Р 7.0.11-2011:
%% 5.3.3 В заключении диссертации излагают итоги выполненного исследования, рекомендации, перспективы дальнейшей разработки темы.
%% 9.2.3 В заключении автореферата диссертации излагают итоги данного исследования, рекомендации и перспективы дальнейшей разработки темы.
%% Поэтому имеет смысл сделать эту часть общей и загрузить из одного файла в автореферат и в диссертацию:

В процессе исследования были проведены многочисленные опыты по определению физических, физико-механических свойств, гранулометрического, минералогического, химического составов грунтов и их закономерностей. 

В ходе проводимого исследования, удалось оценить и сравнить точность определения основных параметров переуплотненных грунтов методами Казагранде и Беккера.

Были выявлены слабые стороны методов определения переуплотнения, а также предложены меры по утранению недостатков этих методов и автоматизации их определения.


Исходя из вышесказанного, можно заявить, что поднятая проблема будет и дальше подвергаться изучению в будущем, так как в процессе поиска материалов были обнаружены и изучены более полусотни способов определения напряжения предуплотнения, однако, рассмотрение этих методов выходило за рамки данной работы.

Их изучению планируется посвятить дипломную работу, а также планируется изучить другие разновидности грунтов и накопить больше материала для статистической обработки.
