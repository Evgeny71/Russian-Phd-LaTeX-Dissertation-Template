\chapter*{Заключение}                       % Заголовок
\addcontentsline{toc}{chapter}{Заключение}  % Добавляем его в оглавление

%% Согласно ГОСТ Р 7.0.11-2011:
%% 5.3.3 В заключении диссертации излагают итоги выполненного исследования, рекомендации, перспективы дальнейшей разработки темы.
%% 9.2.3 В заключении автореферата диссертации излагают итоги данного исследования, рекомендации и перспективы дальнейшей разработки темы.
%% Поэтому имеет смысл сделать эту часть общей и загрузить из одного файла в автореферат и в диссертацию:

В процессе исследования были проведены многочисленные опыты по определению физических, физико-механических свойств, гранулометрического, минералогического, химического составов грунтов, а также выявлены их закономерности. 

В ходе работы, удалось оценить и сравнить точность определения напряжения предуплотненных грунтов $\sigma_p$ методами Казагранде и Беккера. Был проведен социальный эксперимент среди специалистов, который заключался в выявлении субъективного фактора при определении значений, а также в сравнении точности методов определения параметров напряженно-деформируемого состояния грунта. Эксперимент показал, что полученные данные разнятся в большом диапазоне, что является обоснованием для утверждения о важности субъективного фактора при определении данных параметров.

% Также по этим данным обработки можно сказать о том, что наименьшая разница между максимальнымы и минимальными значениями наблюдается при обработке методом Беккера, что отличает его от метода Казагранде. Вероятно, такая зависимость является следствием наибольшего количества сложных графических построений у последнего.
Также по этим данным обработки можно сказать о том, что разница между максимальнымы и минимальными значениями при обработке методом Беккера и методом Казагранде существенно не отличается.

В ходе работы были выявлены слабые стороны методов определения напряжения предуплотнения $\sigma_p$, такие как: сложности с выбором касательной к начальному участку кривой при графических построениях и субъктивный фактор при таком выборе. Под этим фактором подразумевается: определение точки максимальной кривизны, проведение касательных и биссектрис. Своей субъективностью может характеризоваться точка пересечения биссектрисы и продолжения прямолинейного участка кривой.
%Затем предложены меры по устранению недостатков этих методов и автоматизации их определения "--- использование численных методов при обработке полученных результатов, с помощью которых можно снизить вероятность грубых ошибок.


Исходя из вышесказанного, можно заявить, что поднятая проблема будет и дальше подвергаться изучению в будущем, так как в процессе поиска материалов были обнаружены и изучены более полусотни способов определения напряжения предуплотнения, однако, рассмотрение этих методов выходило за рамки данной работы. 
Их изучению планируется посвятить дипломную работу, а также изучить другие разновидности грунтов и накопить больше материала для ведения статистической обработки.
