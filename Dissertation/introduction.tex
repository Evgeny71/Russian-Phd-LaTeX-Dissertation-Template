\chapter*{Введение}                         % Заголовок
\addcontentsline{toc}{chapter}{Введение}    % Добавляем его в оглавление

Все больше в практику проектирования фундаментов сооружений входят расчеты грунтовых оснований с использованием численного моделирования поведения грунтов на базе метода конечных элементов. 
В последнее время они все чаще используется в практике проектирования промышленно-гражданских сооружений. 

Расчеты выполняются в таких программных комплексах, как PLAXIS, MIDAS и др. 
Для повышения достоверности расчетов данные программы наряду со стандартными характеристиками грунтов используют показатели, не отраженные в отечественных нормах. 

В частности, это параметры, характеризующие предварительное напряженное состояние грунта. 
К ним относятся в том числе такие показатели, как:
\begin{itemize}
    \item Напряжение предуплотнения $\sigma_p$ (preconsolidation pressure)
    \item Напряжение переуплотнения $POP$ (preoverburden pressure)
    \item Коэффициент переуплотнения $OCR$ (overconsolidation ratio)
\end{itemize}
Эти параметры нужны чтобы более детально описывать поведение грунтов, вычислительные мощности позволяют. 
Такие модели, как Cam Clay, Soft Soil, Soft Soil Creep, Hardening Soil и прочие, используют эти параметры. 

Переуплотненные грунты широко распространены как на материковой части, так и на шельфе северных морей, в частности на арктическом и дальневосточном шельфе России.
Осадочные грунты, такие как аргиллиты или алевролиты, выше которых отложения отсутствуют или имеют небольшую мощность, являются хорошими примерами переуплотненных грунтов.
Поэтому, при инженерно-геологических изысканиях особенно важно определять параметры грунта, связанные с его предварительным напряженным состоянием.

В настоящее время существует около полусотни  методов исследования  состояния грунта, давления предварительного уплотнения $\sigma_p$. 
Однако, разработчиками ГОСТ 58326-2018 было принято решение включить только два метода: Казагранде и Беккера, которые и будут рассмотрены в данной курсовой работе. 

Основной задачей данной работы является сравнение точности определения $\sigma_p$ методами Беккера и Казагранде, выявление их недостатков на примере испытаний, а также лабораторные определения состава грунта и характеристик основных физических и механических свойств. 
Определение точности этих методов осуществлялось на искусственных образцах глинистых грунтов четвертичного возраста с заданным напряжением переуплотнения. 
В работе игнорируются факторы литогенеза, превращения минералов и прочие, когда образец может получить переуплотненное состояние, минуя большие нагрузки в своем прошлом. 
Рассматривается единственный фактор максимального исторического напряжения.

%ознакомление с наиболее известными зарубежными и отечественными методами определения параметров переуплотнения грунтов,

Работы проводились в ходе инженерно-геологических изысканий под ряд крупных объектов, расположенных в юго-западном районе Москвы НАО, поселение Сосенское, вблизи д. Николо-Хованское, под строящийся жилой комплекс <<Саларьево-парк>> Группы компаний <<ПИК>>. 
Верхняя часть разреза представлена различными стратиграфо-генетическими комплексами пород четвертичной системы. 