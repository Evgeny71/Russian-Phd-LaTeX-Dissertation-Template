\chapter*{Введение}                         % Заголовок
\addcontentsline{toc}{chapter}{Введение}    % Добавляем его в оглавление

\newcommand{\actuality}{}
\newcommand{\progress}{}
\newcommand{\aim}{{\textbf\aimTXT}}
\newcommand{\tasks}{\textbf{\tasksTXT}}
\newcommand{\novelty}{\textbf{\noveltyTXT}}
\newcommand{\influence}{\textbf{\influenceTXT}}
\newcommand{\methods}{\textbf{\methodsTXT}}
\newcommand{\defpositions}{\textbf{\defpositionsTXT}}
\newcommand{\reliability}{\textbf{\reliabilityTXT}}
\newcommand{\probation}{\textbf{\probationTXT}}
\newcommand{\contribution}{\textbf{\contributionTXT}}
\newcommand{\publications}{\textbf{\publicationsTXT}}

\input{common/characteristic} % Характеристика работы по структуре во введении и в автореферате не отличается (ГОСТ Р 7.0.11, пункты 5.3.1 и 9.2.1), потому её загружаем из одного и того же внешнего файла, предварительно задав форму выделения некоторым параметрам

\textbf{Объем и структура работы.} Диссертация состоит из~введения, трёх глав,
заключения и~двух приложений.
%% на случай ошибок оставляю исходный кусок на месте, закомментированным
%Полный объём диссертации составляет  \ref*{TotPages}~страницу
%с~\totalfigures{}~рисунками и~\totaltables{}~таблицами. Список литературы
%содержит \total{citenum}~наименований.
%
Полный объём диссертации составляет
\formbytotal{TotPages}{страниц}{у}{ы}{}, включая
\formbytotal{totalcount@figure}{рисун}{ок}{ка}{ков} и
\formbytotal{totalcount@table}{таблиц}{у}{ы}{}.   Список литературы содержит
\formbytotal{citenum}{наименован}{ие}{ия}{ий}.




\section{Введение}
Все больше в практику проектирования фундаментов сооружений входят расчеты грунтовых оснований с использованием численного моделирования поведения грунтов на базе метода конечных элементов. К ним относятся такие программные комплексы, как PLAXIS, MIDAS и др. Для повышения достоверности расчетов данные программы наряду со стандартными характеристиками грунтов используют показатели, не отраженные в отечественных нормах. В частности, это параметры, характеризующие предварительное напряженное состояние грунта. К ним относятся:
\begin{itemize}
    \item Напряжение предуплотнения $\sigma_p$ (preconsolidation pressure)
    \item Напряжение переуплотнения $POP$ (preoverburden pressure)
    \item Коэффициент переуплотнения $OCR$ (overconsolidation ratio)
\end{itemize}

Переуплотненные грунты широко распространены как на материковой части, так и на шельфе северных морей, в частности на арктическом и дальневосточном шельфе России.
 Осадочные грунты, такие как аргиллиты или алевролиты, выше которых отложения отсутствуют или имеют небольшую мощность, являются хорошими примерами переуплотненных грунтов.
  Поэтому, при инженерно-геологических изысканиях особенно важно определять параметры грунта, связанные с его предварительным напряженным состоянием.


В настоящее время существует довольно много интересных методов исследования  состояния грунта, давления предварительного уплотнения $\sigma_p$, среди которых основопологающими являются методы Казагранде, Беккера, Шмертмана, Пашеко Силва и др, о которых пойдет речь позднее. 

Основной задачей данной работы является ознакомление с наиболее известными зарубежными и отечественными методами определения параметров переуплотнения грунтов, сравнение методов Беккера и Казагранде, выявление их недостатков на примере испытаний глинистых грунтов четвертичного возраста, лабораторные определения состава грунта и характеристик основных физических и механических свойств. 
 
Работы проводились в ходе инженерно-геологических изысканий под ряд крупных объектов, расположенных в юго-западном районе Москвы НАО, поселение Сосенское, вблизи д. Николо-Хованское, где верхняя часть разреза представлена различными стратиграфо-генетическими комплексами пород четвертичной системы. Геологический разрез исследуемой территории был изучен на глубину 35,0 м.

Условия их залегания, распространения, состав, состояние зависят от возраста и генезиса и создают довольно разнородную картину. Изучались дисперсные грунты, представленные глинами и суглинками твердой, полутвердой и тугопластичной консистенции. Испытания грунтов проводились в соответствии с методами, приведенными в действующих нормативных документах(!). Были определены: гранулометрический состав, плотность $\rho$, плотность твердых частиц $\rho_s$, влажность $w_e$, влажность на границе раскатывания и текучести ($w_L$ и $w_p$). Рассчитаны такие параметры для глинистых грунтов, как: плотность сухого грунта $\rho_d$,  коэффициент водонасыщения $S_r$, число пластичности $I_p$ и показатель текучести $I_L$. Коэффициент переуплотнения грунта $OCR$ (как отношение давления предварительного уплотнения ($\sigma_p$) к бытовому давлению в точке отбора образца) и другие характеристики. ("Инженерно-геологические изыскания" Труфанов А.Н. и др. 1997 г.)