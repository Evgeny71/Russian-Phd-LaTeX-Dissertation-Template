\chapter*{Введение}                         % Заголовок
\addcontentsline{toc}{chapter}{Введение}    % Добавляем его в оглавление

Данная работа посвящена изучению параметров, характеризующих предварительное напряженное состояние грунта. Планируется изучить большое количество как современных работ, так и ранних исследований в этой сфере. Такие работы как Казагранде, Беккера, Терцаги и других, имеют огромное значение для современного грунтоведения и механики грунтов по сей день.
В настоящее время в нашей стране только начинается более подробное изучение данного вопроса, о чем говорит введение нового стандарта ГОСТ. 

Все больше в практику проектирования фундаментов сооружений входят расчеты грунтовых оснований с использованием численного моделирования поведения грунтов на базе метода конечных элементов. 
В последнее время они все чаще используются в практике проектирования промышленно-гражданских сооружений. 

Расчеты выполняются в таких программных комплексах, как PLAXIS, MIDAS и др. 
Для повышения достоверности расчетов данные программы наряду со стандартными характеристиками грунтов используют показатели, не отраженные в отечественных нормах. 

В частности, это параметры, характеризующие предварительное напряженное состояние грунта. 
К ним относятся в том числе такие показатели, как:
\begin{itemize}
    \item напряжение предуплотнения $\sigma_p$ (preconsolidation pressure);
    \item напряжение переуплотнения $POP$ (preoverburden pressure);
    \item коэффициент переуплотнения $OCR$ (overconsolidation ratio).
\end{itemize}

Напряжение переуплотнения $POP$ "--- разница между эффективным напряжением предвари­тельного уплотнения $\sigma_p$ и вертикальным эффективным напряжением от собственного веса грунта $\sigma_0$.
\begin{equation}
    \label{eq:pop}
    POP = \sigma_p - \sigma_0
\end{equation}
Коэффициент переуплотнения $OCR$ "---  отношение эффективного напряжения предваритель­ного уплотнения $\sigma_p$ к вертикальному эффективному напряжению от собственного веса грунта $\sigma_0$.
\begin{equation}
    \label{eq:ocr}
    OCR = \frac{\sigma_p}{\sigma_0}
\end{equation}

Эти параметры нужны чтобы более детально описывать поведение грунтов, вычислительные мощности позволяют. 
Такие модели, как Cam Clay, Soft Soil, Soft Soil Creep, Hardening Soil и прочие, используют эти параметры. 

Переуплотненные грунты широко распространены как на материковой части, так и на шельфе северных морей, в частности на арктическом и дальневосточном шельфе России.
Осадочные грунты, такие как аргиллиты или алевролиты, выше которых отложения отсутствуют или имеют небольшую мощность, являются хорошими примерами переуплотненных грунтов.
Поэтому, при инженерно-геологических изысканиях особенно важно определять параметры грунта, связанные с его предварительным напряженным состоянием.

В настоящее время существует около полусотни  методов определения напряжения предуплотнения $\sigma_p$. (Крамаренко, 2014) \cite{kram2014}
Однако, разработчиками ГОСТ 58326-2018 было принято решение включить только два метода: Казагранде и Беккера, которые и будут рассмотрены в данной курсовой работе. 

Основной целью данной работы является сравнение точности определения $\sigma_p$ и, соответственно, $OSR$ методами Беккера и Казагранде, выявление их недостатков на примере испытаний, а также лабораторные определения состава грунта и характеристик основных физических и физико-механических свойств. 
Определение точности этих методов будет осуществляться на искусственных образцах глинистых грунтов четвертичного возраста с заданным напряжением переуплотнения. 
В работе будет рассматриваться единственный фактор максимального исторического напряжения. Прочие факторы, при которых грунт может получить переуплотненное состояние, минуя большие нагрузки в своем прошлом, такие как литогенез, превращения минералов, выщелачивание и прочие не будут учитываться.

\textbf{Цели.}
Для достижения данной цели будут проведены следующие испытания:
\begin{itemize}
    \item отбор проб и пробоподготовка; 
    \item определение твердой компоненты грунта;
    \item химический анализ образцов;
    \item гранулометрический анализ образцов;
    \item рентгено-структурный анализ образцов;
    \item подготовка к компрессионным испытаниям и их осуществление;
    \item анализ и обработка полученных данных методами Казагранде и Беккера;
    \item оптимизация и автоматизация процессов определения напряжения предуплотнения грунтов $\sigma_p$;
    \item изучение международных стандартов.
\end{itemize}

%ознакомление с наиболее известными зарубежными и отечественными методами определения параметров переуплотнения грунтов,

Образцы для исследований были отобраны в ходе инженерно-геологических изысканий под ряд крупных объектов, расположенных в юго-западном районе Москвы НАО, поселение Сосенское, вблизи д. Николо-Хованское, под строящийся жилой комплекс <<Саларьево-парк>> группы компаний <<ПИК>>. 
 

\textbf{Благодарности.}

Особую благодарность хочу выразить: моему научному руководителю В.\;В.\;Шаниной, а также главному специалисту лаборатории ООО <<ГеоГрадСтрой>> В.\;В.\;Матвееву за бесценную помощь и поддержку на всех этапах исследований; сотруднику лаборатории механики грунтов В.\;С.\;Чочиава-Степаненко за помощь в компрессионных испытаниях; В.\;В.\;Крупской и С.\;А.\;Гараниной за выполнение и обработку данных рентгено-структурного анализа.
Испытания планируется провести с октября 2019 г. по март 2020 г. Обработка будет выполняться по мере завершения испытаний.