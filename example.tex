%&preformat-disser
\RequirePackage[l2tabu,orthodox]{nag} % Раскомментировав, можно в логе получать рекомендации относительно правильного использования пакетов и предупреждения об устаревших и нерекомендуемых пакетах
% Формат А4, 14pt (ГОСТ Р 7.0.11-2011, 5.3.6)
\documentclass[a4paper,14pt,oneside,openany]{memoir}

%%%%%%%%%%%%%%%%%%%%%%%%%%%%%%%%%%%%%%%%%%%%%%%%%%%%%%
%%%% Файл упрощённых настроек шаблона диссертации %%%%
%%%%%%%%%%%%%%%%%%%%%%%%%%%%%%%%%%%%%%%%%%%%%%%%%%%%%%

%%% Инициализирование переменных, не трогать!  %%%
\newcounter{intvl}
\newcounter{otstup}
\newcounter{contnumeq}
\newcounter{contnumfig}
\newcounter{contnumtab}
\newcounter{pgnum}
\newcounter{chapstyle}
\newcounter{headingdelim}
\newcounter{headingalign}
\newcounter{headingsize}
\newcounter{tabcap}
\newcounter{tablaba}
\newcounter{tabtita}
%%%%%%%%%%%%%%%%%%%%%%%%%%%%%%%%%%%%%%%%%%%%%%%%%%%%%%

%%% Область упрощённого управления оформлением %%%

%% Интервал между заголовками и между заголовком и текстом %%
% Заголовки отделяют от текста сверху и снизу
% тремя интервалами (ГОСТ Р 7.0.11-2011, 5.3.5)
\setcounter{intvl}{3}               % Коэффициент кратности к размеру шрифта

%% Отступы у заголовков в тексте %%
\setcounter{otstup}{0}              % 0 --- без отступа; 1 --- абзацный отступ

%% Нумерация формул, таблиц и рисунков %%
% Нумерация формул
\setcounter{contnumeq}{1}   % 0 --- пораздельно (во введении подряд,
                            %       без номера раздела);
                            % 1 --- сквозная нумерация по всей диссертации
% Нумерация рисунков
\setcounter{contnumfig}{1}  % 0 --- пораздельно (во введении подряд,
                            %       без номера раздела);
                            % 1 --- сквозная нумерация по всей диссертации
% Нумерация таблиц
\setcounter{contnumtab}{1}  % 0 --- пораздельно (во введении подряд,
                            %       без номера раздела);
                            % 1 --- сквозная нумерация по всей диссертации

%% Оглавление %%
\setcounter{pgnum}{1}       % 0 --- номера страниц никак не обозначены;
                            % 1 --- Стр. над номерами страниц (дважды
                            %       компилировать после изменения настройки)
\settocdepth{subsection}    % до какого уровня подразделов выносить в оглавление
\setsecnumdepth{subsection} % до какого уровня нумеровать подразделы


%% Текст и форматирование заголовков %%
\setcounter{chapstyle}{1}     % 0 --- разделы только под номером;
                              % 1 --- разделы с названием "Глава" перед номером
\setcounter{headingdelim}{1}  % 0 --- номер отделен пропуском в 1em или \quad;
                              % 1 --- номера разделов и приложений отделены
                              %       точкой с пробелом, подразделы пропуском
                              %       без точки;
                              % 2 --- номера разделов, подразделов и приложений
                              %       отделены точкой с пробелом.

%% Выравнивание заголовков в тексте %%
\setcounter{headingalign}{0}  % 0 --- по центру;
                              % 1 --- по левому краю

%% Размеры заголовков в тексте %%
\setcounter{headingsize}{0}   % 0 --- по ГОСТ, все всегда 14 пт;
                              % 1 --- пропорционально изменяющийся размер
                              %       в зависимости от базового шрифта

%% Подпись таблиц %%
\setcounter{tabcap}{0}  % 0 --- по ГОСТ, номер таблицы и название разделены
                        %       тире, выровнены по левому краю, при
                        %       необходимостина нескольких строках;
                        % 1 --- подпись таблицы не по ГОСТ, на двух и более
                        %       строках, дальнейшие настройки:
%Выравнивание первой строки, с подписью и номером
\setcounter{tablaba}{2} % 0 --- по левому краю;
                        % 1 --- по центру;
                        % 2 --- по правому краю
%Выравнивание строк с самим названием таблицы
\setcounter{tabtita}{1} % 0 --- по левому краю;
                        % 1 --- по центру;
                        % 2 --- по правому краю
%Разделитель записи «Таблица #» и названия таблицы
\newcommand{\tablabelsep}{ }

%% Подпись рисунков %%
%Разделитель записи «Рисунок #» и названия рисунка
\newcommand{\figlabelsep}{~\cyrdash\ }  % (ГОСТ 2.105, 4.3.1)
                                        % "--- здесь не работает

%%% Цвета гиперссылок %%%
% Latex color definitions: http://latexcolor.com/
\definecolor{linkcolor}{rgb}{0.9,0,0}
\definecolor{citecolor}{rgb}{0,0.6,0}
\definecolor{urlcolor}{rgb}{0,0,1}
%\definecolor{linkcolor}{rgb}{0,0,0} %black
%\definecolor{citecolor}{rgb}{0,0,0} %black
%\definecolor{urlcolor}{rgb}{0,0,0} %black
            % общие настройки шаблона
%%% Проверка используемого TeX-движка %%%
\newif\ifxetexorluatex   % определяем новый условный оператор (http://tex.stackexchange.com/a/47579)
\ifxetex
    \xetexorluatextrue
\else
    \ifluatex
        \xetexorluatextrue
    \else
        \xetexorluatexfalse
    \fi
\fi

\newif\ifsynopsis           % Условие, проверяющее, что документ --- автореферат

\usepackage{etoolbox}[2015/08/02]   % Для продвинутой проверки разных условий
\providebool{presentation}

\usepackage{comment}    % Позволяет убирать блоки текста (добавляет
                        % окружение comment и команду \excludecomment)

%%% Поля и разметка страницы %%%
\usepackage{pdflscape}  % Для включения альбомных страниц
\usepackage{geometry}   % Для последующего задания полей

%%% Математические пакеты %%%
\usepackage{amsthm,amsmath,amscd}   % Математические дополнения от AMS
\usepackage{amsfonts,amssymb}       % Математические дополнения от AMS
\usepackage{mathtools}              % Добавляет окружение multlined
\usepackage{xfrac}                  % Красивые дроби
\usepackage[
    locale = DE,
    list-separator       = {;\,},
    list-final-separator = {;\,},
    list-pair-separator  = {;\,},
    range-phrase={\text{\ensuremath{-}}},
    % quotient-mode        = fraction, % красивые дроби могут не соответствовать ГОСТ
    fraction-function    = \sfrac,
    separate-uncertainty,
    ]{siunitx}                      % Размерности SI
\sisetup{inter-unit-product = \ensuremath{{}\cdot{}}}

% Кириллица в нумерации subequations
% Для правильной работы требуется выполнение сразу после загрузки пакетов
\patchcmd{\subequations}{\def\theequation{\theparentequation\alph{equation}}}
{\def\theequation{\theparentequation\asbuk{equation}}}
{\typeout{subequations patched}}{\typeout{subequations not patched}}

%%%% Установки для размера шрифта 14 pt %%%%
%% Формирование переменных и констант для сравнения (один раз для всех подключаемых файлов)%%
%% должно располагаться до вызова пакета fontspec или polyglossia, потому что они сбивают его работу
\newlength{\curtextsize}
\newlength{\bigtextsize}
\setlength{\bigtextsize}{13.9pt}

\makeatletter
%\show\f@size    % неплохо для отслеживания, но вызывает стопорение процесса,
                 % если документ компилируется без команды  -interaction=nonstopmode
\setlength{\curtextsize}{\f@size pt}
\makeatother

%%% Кодировки и шрифты %%%
\ifxetexorluatex
    \PassOptionsToPackage{no-math}{fontspec}    % https://tex.stackexchange.com/a/26295/104425
    \usepackage{polyglossia}[2014/05/21]        % Поддержка многоязычности
                                        % (fontspec подгружается автоматически)
\else
   %%% Решение проблемы копирования текста в буфер кракозябрами
    \ifnumequal{\value{usealtfont}}{0}{}{
        \input glyphtounicode.tex
        \input glyphtounicode-cmr.tex %from pdfx package
        \pdfgentounicode=1
    }
    \usepackage{cmap}   % Улучшенный поиск русских слов в полученном pdf-файле
    \ifnumequal{\value{usealtfont}}{2}{}{
        \defaulthyphenchar=127  % Если стоит до fontenc, то переносы
                                % не впишутся в выделяемый текст при
                                % копировании его в буфер обмена
    }
    \usepackage{textcomp}
    \usepackage[T1,T2A]{fontenc}                    % Поддержка русских букв
    \ifnumequal{\value{usealtfont}}{1}{% Используется pscyr, при наличии
        \IfFileExists{pscyr.sty}{\usepackage{pscyr}}{}  % Подключение pscyr
    }{}
    \usepackage[utf8]{inputenc}[2014/04/30]         % Кодировка utf8
    \usepackage[english, russian]{babel}[2014/03/24]% Языки: русский, английский
    \makeatletter\AtBeginDocument{\let\@elt\relax}\makeatother % babel 3.40 fix
    \ifnumequal{\value{usealtfont}}{2}{
        % http://dxdy.ru/post1238763.html#p1238763
        \usepackage[scaled=0.960]{XCharter}[2017/12/19] % Подключение русифицированных шрифтов XCharter
        \usepackage[charter, vvarbb, scaled=1.048]{newtxmath}[2017/12/14]
        \ifpresentation
        \else
            \setDisplayskipStretch{-0.078}
        \fi
    }{}
\fi

%%% Оформление абзацев %%%
\usepackage{indentfirst}                            % Красная строка

%%% Цвета %%%
\ifpresentation
\else
    \usepackage[dvipsnames, table, hyperref]{xcolor} % Совместимо с tikz
\fi

%%% Таблицы %%%
\usepackage{longtable,ltcaption} % Длинные таблицы
\usepackage{multirow,makecell}   % Улучшенное форматирование таблиц
\usepackage{tabu, tabulary}      % таблицы с автоматически подбирающейся
                                 % шириной столбцов (tabu обязательно
                                 % до hyperref вызывать)
\usepackage{threeparttable}      % автоматический подгон ширины подписи таблицы

%%% Общее форматирование
\usepackage{soulutf8}% Поддержка переносоустойчивых подчёркиваний и зачёркиваний
\usepackage{icomma}  % Запятая в десятичных дробях

%%% Оптимизация расстановки переносов и длины последней строки абзаца
\IfFileExists{impnattypo.sty}{% проверка установленности пакета impnattypo
    \ifluatex
        \ifnumequal{\value{draft}}{1}{% Черновик
            \usepackage[hyphenation, lastparline, nosingleletter, homeoarchy,
            rivers, draft]{impnattypo}
        }{% Чистовик
            \usepackage[hyphenation, lastparline, nosingleletter]{impnattypo}
        }
    \else
        \usepackage[hyphenation, lastparline]{impnattypo}
    \fi
}{}

%% Векторная графика

\usepackage{tikz}                   % Продвинутый пакет векторной графики
\usetikzlibrary{chains}             % Для примера tikz рисунка
\usetikzlibrary{shapes.geometric}   % Для примера tikz рисунка
\usetikzlibrary{shapes.symbols}     % Для примера tikz рисунка
\usetikzlibrary{arrows}             % Для примера tikz рисунка
\usetikzlibrary{plotmarks}

\usepackage{pgfplots}
\usepgfplotslibrary{ternary}


%%% Гиперссылки %%%
\usepackage{hyperref}[2012/11/06]

%%% Изображения %%%
\usepackage{graphicx}[2014/04/25]   % Подключаем пакет работы с графикой

%%% Счётчики %%%
\usepackage[figure,table]{totalcount}   % Счётчик рисунков и таблиц
\usepackage{totcount}   % Пакет создания счётчиков на основе последнего номера
                        % подсчитываемого элемента (может требовать дважды
                        % компилировать документ)
\usepackage{totpages}   % Счётчик страниц, совместимый с hyperref (ссылается
                        % на номер последней страницы). Желательно ставить
                        % последним пакетом в преамбуле

%%% Продвинутое управление групповыми ссылками (пока только формулами) %%%
\ifpresentation
\else
    \usepackage[russian]{cleveref} % cleveref имеет сложности со считыванием
    % языка из babel. Такое решение русификации вывода выбрано вместо
    % определения в documentclass из опасности что-то лишнее передать во все
    % остальные пакеты, включая библиографию.
    \creflabelformat{equation}{#2#1#3} % Формат по умолчанию ставил круглые
    % скобки вокруг каждого номера ссылки, теперь просто номера ссылок без
    % какого-либо дополнительного оформления
    \crefrangelabelformat{equation}{#3#1#4\cyrdash#5#2#6} % Интервалы в русском
    % языке принято делать через тире, если иное не оговорено

    % решение проблемы с "и" в \labelcref
    % https://tex.stackexchange.com/a/455124/104425
    \ifxetexorluatex
        \DeclareTextSymbol{\cyri}\UnicodeEncodingName{"0438} % и
    \fi

    % Добавление возможности использования пробелов в \labelcref
    % https://tex.stackexchange.com/a/340502/104425
    \usepackage{kvsetkeys}
    \makeatletter
    \let\org@@cref\@cref
    \renewcommand*{\@cref}[2]{%
        \edef\process@me{%
            \noexpand\org@@cref{#1}{\zap@space#2 \@empty}%
        }\process@me
    }
    \makeatother

    \newcommand{\eqrefs}[1]{(\labelcref{#1})}
    \newcommand{\refs}[1]{\labelcref{#1}}
\fi

\ifnumequal{\value{draft}}{1}{% Черновик
    \usepackage[firstpage]{draftwatermark}
    \SetWatermarkText{DRAFT}
    \SetWatermarkFontSize{14pt}
    \SetWatermarkScale{15}
    \SetWatermarkAngle{45}
}{}

%%% Исправление положения якорей подписей (под)рисунков %%%
% Без hypcap и патча, при клике по ссылке на подрисунок, просмотрщик pdf прыгает "к подписи" а не "к рисунку".
% Подробнее: https://github.com/AndreyAkinshin/Russian-Phd-LaTeX-Dissertation-Template/issues/238
% (!) Даже с патчем, если мешать в одной фиге разные типы подфиг (subbottom и subcaption) - ссылки всё равно будут работать неправильно  (см. https://www.overleaf.com/read/czmbmmtnqrrg ).
\ifpresentation
\else
    \usepackage[all]{hypcap}

    \makeatletter
    \ltx@ifclasslater{memoir}{2018/12/13}{
        % Предполагается, что в следующей версии класс будет исправлен
        \typeout{Assuming this version of memoir is free from the jumping-to-caption bug.}
    }{
        \usepackage{xpatch}

        \newcommand\mem@step@subcounter{\refstepcounter{sub\@captype}\@contkeep}

        \xpatchcmd{\@memsubbody}%
        {\refstepcounter{sub\@captype}\@contkeep}% search pattern
        {}% replacement
        {\typeout{@memsubbody is patched}}%
        {\typeout{@memsubbody is NOT patched}}%

        \xpatchcmd{\@memcontsubbody}%
        {\refstepcounter{sub\@captype}\@contkeep}% pattern
        {}% replacement
        {\typeout{@memcontsubbody is patched}}%
        {\typeout{@memcontsubbody is NOT patched}}%

        \xpatchcmd{\@memsubfloat}%
        {\vbox\bgroup}% search pattern
        {\vbox\bgroup\mem@step@subcounter}% replacement
        {\typeout{@memsubfloat patch is ok}}%
        {\typeout{@memsubfloat patch is NOT ok}}%

        \xpatchcmd{\subcaption}%
        {\refstepcounter{sub\@captype}}% search pattern
        {\H@refstepcounter{sub\@captype}}% replacement
        {\typeout{subcaption second patch is ok}}%
        {\typeout{subcaption second patch is NOT ok}}%
    }
    \makeatother
\fi

%%% Цитата, не приводимая в автореферате:
% возможно, актуальна только для biblatex
%\newcommand{\citeinsynopsis}[1]{\ifsynopsis\else ~\cite{#1} \fi}

% если текущий процесс запущен библиотекой tikz-external, то прекомпиляция должна быть включена
\ifdefined\tikzexternalrealjob
    \setcounter{imgprecompile}{1}
\fi

\ifnumequal{\value{imgprecompile}}{1}{% Только если у нас включена предкомпиляция
    \usetikzlibrary{external}   % подключение возможности предкомпиляции
    \tikzexternalize[prefix=images/cache/] % activate! % здесь можно указать отдельную папку для скомпилированных файлов
    \ifxetex
        \tikzset{external/up to date check={diff}}
    \fi
}{}
         % Пакеты общие для диссертации и автореферата
\synopsisfalse                      % Этот документ --- не автореферат
\input{Dissertation/dispackages}    % Пакеты для диссертации
\usepackage{fr-longtable}    %ради \endlasthead

% Листинги с исходным кодом программ
\usepackage{fancyvrb}
\usepackage{listings}
\lccode`\~=0\relax %Без этого хака из-за особенностей пакета listings перестают работать конструкции с \MakeLowercase и т. п. в (xe|lua)latex

% Русская традиция начертания греческих букв
\usepackage{upgreek} % прямые греческие ради русской традиции

%%% Микротипографика
%\ifnumequal{\value{draft}}{0}{% Только если у нас режим чистовика
%    \usepackage[final, babel, shrink=45]{microtype}[2016/05/14] % улучшает представление букв и слов в строках, может помочь при наличии отдельно висящих слов
%}{}

% Отметка о версии черновика на каждой странице
% Чтобы работало надо в своей локальной копии по инструкции
% https://www.ctan.org/pkg/gitinfo2 создать небходимые файлы в папке
% ./git/hooks
% If you’re familiar with tweaking git, you can probably work it out for
% yourself. If not, I suggest you follow these steps:
% 1. First, you need a git repository and working tree. For this example,
% let’s suppose that the root of the working tree is in ~/compsci
% 2. Copy the file post-xxx-sample.txt (which is in the same folder of
% your TEX distribution as this pdf) into the git hooks directory in your
% working copy. In our example case, you should end up with a file called
% ~/compsci/.git/hooks/post-checkout
% 3. If you’re using a unix-like system, don’t forget to make the file executable.
% Just how you do this is outside the scope of this manual, but one
% possible way is with commands such as this:
% chmod g+x post-checkout.
% 4. Test your setup with “git checkout master” (or another suitable branch
% name). This should generate copies of gitHeadInfo.gin in the directories
% you intended.
% 5. Now make two more copies of this file in the same directory (hooks),
% calling them post-commit and post-merge, and you’re done. As before,
% users of unix-like systems should ensure these files are marked as
% executable.
\ifnumequal{\value{draft}}{1}{% Черновик
   \IfFileExists{.git/gitHeadInfo.gin}{
      \usepackage[mark,pcount]{gitinfo2}
      \renewcommand{\gitMark}{rev.\gitAbbrevHash\quad\gitCommitterEmail\quad\gitAuthorIsoDate}
      \renewcommand{\gitMarkFormat}{\rmfamily\color{Gray}\small\bfseries}
   }{}
}{}

\usepackage{rotating}
\usepackage{pgfplotstable}
   % Пакеты для специфических пользовательских задач

%%%%%%%%%%%%%%%%%%%%%%%%%%%%%%%%%%%%%%%%%%%%%%%%%%%%%%
%%%% Файл упрощённых настроек шаблона диссертации %%%%
%%%%%%%%%%%%%%%%%%%%%%%%%%%%%%%%%%%%%%%%%%%%%%%%%%%%%%

%%% Инициализирование переменных, не трогать!  %%%
\newcounter{intvl}
\newcounter{otstup}
\newcounter{contnumeq}
\newcounter{contnumfig}
\newcounter{contnumtab}
\newcounter{pgnum}
\newcounter{chapstyle}
\newcounter{headingdelim}
\newcounter{headingalign}
\newcounter{headingsize}
\newcounter{tabcap}
\newcounter{tablaba}
\newcounter{tabtita}
%%%%%%%%%%%%%%%%%%%%%%%%%%%%%%%%%%%%%%%%%%%%%%%%%%%%%%

%%% Область упрощённого управления оформлением %%%

%% Интервал между заголовками и между заголовком и текстом %%
% Заголовки отделяют от текста сверху и снизу
% тремя интервалами (ГОСТ Р 7.0.11-2011, 5.3.5)
\setcounter{intvl}{3}               % Коэффициент кратности к размеру шрифта

%% Отступы у заголовков в тексте %%
\setcounter{otstup}{0}              % 0 --- без отступа; 1 --- абзацный отступ

%% Нумерация формул, таблиц и рисунков %%
% Нумерация формул
\setcounter{contnumeq}{1}   % 0 --- пораздельно (во введении подряд,
                            %       без номера раздела);
                            % 1 --- сквозная нумерация по всей диссертации
% Нумерация рисунков
\setcounter{contnumfig}{1}  % 0 --- пораздельно (во введении подряд,
                            %       без номера раздела);
                            % 1 --- сквозная нумерация по всей диссертации
% Нумерация таблиц
\setcounter{contnumtab}{1}  % 0 --- пораздельно (во введении подряд,
                            %       без номера раздела);
                            % 1 --- сквозная нумерация по всей диссертации

%% Оглавление %%
\setcounter{pgnum}{1}       % 0 --- номера страниц никак не обозначены;
                            % 1 --- Стр. над номерами страниц (дважды
                            %       компилировать после изменения настройки)
\settocdepth{subsection}    % до какого уровня подразделов выносить в оглавление
\setsecnumdepth{subsection} % до какого уровня нумеровать подразделы


%% Текст и форматирование заголовков %%
\setcounter{chapstyle}{1}     % 0 --- разделы только под номером;
                              % 1 --- разделы с названием "Глава" перед номером
\setcounter{headingdelim}{1}  % 0 --- номер отделен пропуском в 1em или \quad;
                              % 1 --- номера разделов и приложений отделены
                              %       точкой с пробелом, подразделы пропуском
                              %       без точки;
                              % 2 --- номера разделов, подразделов и приложений
                              %       отделены точкой с пробелом.

%% Выравнивание заголовков в тексте %%
\setcounter{headingalign}{0}  % 0 --- по центру;
                              % 1 --- по левому краю

%% Размеры заголовков в тексте %%
\setcounter{headingsize}{0}   % 0 --- по ГОСТ, все всегда 14 пт;
                              % 1 --- пропорционально изменяющийся размер
                              %       в зависимости от базового шрифта

%% Подпись таблиц %%
\setcounter{tabcap}{0}  % 0 --- по ГОСТ, номер таблицы и название разделены
                        %       тире, выровнены по левому краю, при
                        %       необходимостина нескольких строках;
                        % 1 --- подпись таблицы не по ГОСТ, на двух и более
                        %       строках, дальнейшие настройки:
%Выравнивание первой строки, с подписью и номером
\setcounter{tablaba}{2} % 0 --- по левому краю;
                        % 1 --- по центру;
                        % 2 --- по правому краю
%Выравнивание строк с самим названием таблицы
\setcounter{tabtita}{1} % 0 --- по левому краю;
                        % 1 --- по центру;
                        % 2 --- по правому краю
%Разделитель записи «Таблица #» и названия таблицы
\newcommand{\tablabelsep}{ }

%% Подпись рисунков %%
%Разделитель записи «Рисунок #» и названия рисунка
\newcommand{\figlabelsep}{~\cyrdash\ }  % (ГОСТ 2.105, 4.3.1)
                                        % "--- здесь не работает

%%% Цвета гиперссылок %%%
% Latex color definitions: http://latexcolor.com/
\definecolor{linkcolor}{rgb}{0.9,0,0}
\definecolor{citecolor}{rgb}{0,0.6,0}
\definecolor{urlcolor}{rgb}{0,0,1}
%\definecolor{linkcolor}{rgb}{0,0,0} %black
%\definecolor{citecolor}{rgb}{0,0,0} %black
%\definecolor{urlcolor}{rgb}{0,0,0} %black
      % Упрощённые настройки шаблона

% Новые переменные, которые могут использоваться во всём проекте
% ГОСТ 7.0.11-2011
% 9.2 Оформление текста автореферата диссертации
% 9.2.1 Общая характеристика работы включает в себя следующие основные структурные
% элементы:
% актуальность темы исследования;
\newcommand{\actualityTXT}{Актуальность темы.}
% степень ее разработанности;
\newcommand{\progressTXT}{Степень разработанности темы.}
% цели и задачи;
\newcommand{\aimTXT}{Целью}
\newcommand{\tasksTXT}{задачи}
% научную новизну;
\newcommand{\noveltyTXT}{Научная новизна:}
% теоретическую и практическую значимость работы;
%\newcommand{\influenceTXT}{Теоретическая и практическая значимость}
% или чаще используют просто
\newcommand{\influenceTXT}{Практическая значимость}
% методологию и методы исследования;
\newcommand{\methodsTXT}{Методология и методы исследования.}
% положения, выносимые на защиту;
\newcommand{\defpositionsTXT}{Основные положения, выносимые на~защиту:}
% степень достоверности и апробацию результатов.
\newcommand{\reliabilityTXT}{Достоверность}
\newcommand{\probationTXT}{Апробация работы.}

\newcommand{\contributionTXT}{Личный вклад.}
\newcommand{\publicationsTXT}{Публикации.}


%%% Заголовки библиографии:

% для автореферата:
\newcommand{\bibtitleauthor}{Публикации автора по теме диссертации}

% для стиля библиографии `\insertbiblioauthorgrouped`
\newcommand{\bibtitleauthorvak}{В изданиях из списка ВАК РФ}
\newcommand{\bibtitleauthorscopus}{В изданиях, входящих в международную базу цитирования Scopus}
\newcommand{\bibtitleauthorwos}{В изданиях, входящих в международную базу цитирования Web of Science}
\newcommand{\bibtitleauthorother}{В прочих изданиях}
\newcommand{\bibtitleauthorconf}{В сборниках трудов конференций}

% для стиля библиографии `\insertbiblioauthorimportant`:
\newcommand{\bibtitleauthorimportant}{Наиболее значимые \protect\MakeLowercase\bibtitleauthor}

% для списка литературы в диссертации и списка чужих работ в автореферате:
\newcommand{\bibtitlefull}{Список литературы} % (ГОСТ Р 7.0.11-2011, 4)
\newcommand{\q}[3]{#1\:#2\:\textit{#3}}         % Новые переменные, для всего проекта

\input{common/data}             % Основные сведения
\input{common/fonts}            % Определение шрифтов (частичное)
%%% Шаблон %%%
\DeclareRobustCommand{\fixme}{\textcolor{red}}  % решаем проблему превращения
                                % названия цвета в результате \MakeUppercase,
                                % http://tex.stackexchange.com/a/187930,
                                % \DeclareRobustCommand protects \fixme
                                % from expanding inside \MakeUppercase
\AtBeginDocument{%
    \setlength{\parindent}{2.5em}                   % Абзацный отступ. Должен быть одинаковым по всему тексту и равен пяти знакам (ГОСТ Р 7.0.11-2011, 5.3.7).
}

%%% Подписи %%%
\setlength{\abovecaptionskip}{0pt}   % Отбивка над подписью
\setlength{\belowcaptionskip}{0pt}   % Отбивка под подписью
\captionwidth{\linewidth}
\normalcaptionwidth

%%% Таблицы %%%
\ifnumequal{\value{tabcap}}{0}{%
    \newcommand{\tabcapalign}{\raggedright}  % по левому краю страницы или аналога parbox
    \renewcommand{\tablabelsep}{~\cyrdash\ } % тире как разделитель идентификатора с номером от наименования
    \newcommand{\tabtitalign}{}
}{%
    \ifnumequal{\value{tablaba}}{0}{%
        \newcommand{\tabcapalign}{\raggedright}  % по левому краю страницы или аналога parbox
    }{}

    \ifnumequal{\value{tablaba}}{1}{%
        \newcommand{\tabcapalign}{\centering}    % по центру страницы или аналога parbox
    }{}

    \ifnumequal{\value{tablaba}}{2}{%
        \newcommand{\tabcapalign}{\raggedleft}   % по правому краю страницы или аналога parbox
    }{}

    \ifnumequal{\value{tabtita}}{0}{%
        \newcommand{\tabtitalign}{\par\raggedright}  % по левому краю страницы или аналога parbox
    }{}

    \ifnumequal{\value{tabtita}}{1}{%
        \newcommand{\tabtitalign}{\par\centering}    % по центру страницы или аналога parbox
    }{}

    \ifnumequal{\value{tabtita}}{2}{%
        \newcommand{\tabtitalign}{\par\raggedleft}   % по правому краю страницы или аналога parbox
    }{}
}

\precaption{\tabcapalign} % всегда идет перед подписью или \legend
\captionnamefont{\normalfont\small} % Шрифт надписи «Таблица #»; также определяет шрифт у \legend
\captiondelim{\tablabelsep} % разделитель идентификатора с номером от наименования
\captionstyle[\tabtitalign]{\tabtitalign}
\captiontitlefont{\normalfont\small} % Шрифт с текстом подписи

%%% Рисунки %%%
\setfloatadjustment{figure}{%
    \setlength{\abovecaptionskip}{0pt}   % Отбивка над подписью
    \setlength{\belowcaptionskip}{0pt}   % Отбивка под подписью
    \precaption{} % всегда идет перед подписью или \legend
    \captionnamefont{\normalfont\small} % Шрифт надписи «Рисунок #»; также определяет шрифт у \legend
    \captiondelim{\figlabelsep} % разделитель идентификатора с номером от наименования
    \captionstyle[\centering]{\centering} % Центрирование подписей, заданных командой \caption и \legend
    \captiontitlefont{\normalfont\small} % Шрифт с текстом подписи
    \postcaption{} % всегда идет после подписи или \legend, и с новой строки
}

%%% Подписи подрисунков %%%
\newsubfloat{figure} % Включает возможность использовать подрисунки у окружений figure
\renewcommand{\thesubfigure}{\asbuk{subfigure}}           % Буквенные номера подрисунков
\subcaptionsize{\small} % Шрифт подписи названий подрисунков (не отличается от основного)
\subcaptionlabelfont{\normalfont}
\subcaptionfont{\!\!) \normalfont} % Вот так тут добавили скобку после буквы.
\subcaptionstyle{\centering}
%\subcaptionsize{\fontsize{12pt}{13pt}\selectfont} % объявляем шрифт 12pt для использования в подписях, тут же надо интерлиньяж объявлять, если не наследуется

%%% Настройки гиперссылок %%%
\ifluatex
    \hypersetup{
        unicode,                % Unicode encoded PDF strings
    }
\fi

\hypersetup{
    linktocpage=true,           % ссылки с номера страницы в оглавлении, списке таблиц и списке рисунков
%    linktoc=all,                % both the section and page part are links
%    pdfpagelabels=false,        % set PDF page labels (true|false)
    plainpages=false,           % Forces page anchors to be named by the Arabic form  of the page number, rather than the formatted form
    colorlinks,                 % ссылки отображаются раскрашенным текстом, а не раскрашенным прямоугольником, вокруг текста
    linkcolor={linkcolor},      % цвет ссылок типа ref, eqref и подобных
    citecolor={citecolor},      % цвет ссылок-цитат
    urlcolor={urlcolor},        % цвет гиперссылок
%    hidelinks,                  % Hide links (removing color and border)
    pdftitle={\thesisTitle},    % Заголовок
    pdfauthor={\thesisAuthor},  % Автор
    pdfsubject={\thesisSpecialtyNumber\ \thesisSpecialtyTitle},      % Тема
%    pdfcreator={Создатель},     % Создатель, Приложение
%    pdfproducer={Производитель},% Производитель, Производитель PDF
    pdfkeywords={\keywords},    % Ключевые слова
    pdflang={ru},
}
\ifnumequal{\value{draft}}{1}{% Черновик
    \hypersetup{
        draft,
    }
}{}

%%% Списки %%%
% Используем короткое тире (endash) для ненумерованных списков (ГОСТ 2.105-95, пункт 4.1.7, требует дефиса, но так лучше смотрится)
\renewcommand{\labelitemi}{\normalfont\bfseries{--}}

% Перечисление строчными буквами латинского алфавита (ГОСТ 2.105-95, 4.1.7)
%\renewcommand{\theenumi}{\alph{enumi}}
%\renewcommand{\labelenumi}{\theenumi)}

% Перечисление строчными буквами русского алфавита (ГОСТ 2.105-95, 4.1.7)
\makeatletter
\AddEnumerateCounter{\asbuk}{\russian@alph}{щ}      % Управляем списками/перечислениями через пакет enumitem, а он 'не знает' про asbuk, потому 'учим' его
\makeatother
%\renewcommand{\theenumi}{\asbuk{enumi}} %первый уровень нумерации
%\renewcommand{\labelenumi}{\theenumi)} %первый уровень нумерации
\renewcommand{\theenumii}{\asbuk{enumii}} %второй уровень нумерации
\renewcommand{\labelenumii}{\theenumii)} %второй уровень нумерации
\renewcommand{\theenumiii}{\arabic{enumiii}} %третий уровень нумерации
\renewcommand{\labelenumiii}{\theenumiii)} %третий уровень нумерации

\setlist{nosep,%                                    % Единый стиль для всех списков (пакет enumitem), без дополнительных интервалов.
    labelindent=\parindent,leftmargin=*%            % Каждый пункт, подпункт и перечисление записывают с абзацного отступа (ГОСТ 2.105-95, 4.1.8)
}

%%% Правильная нумерация приложений, рисунков и формул %%%
%% По ГОСТ 2.105, п. 4.3.8 Приложения обозначают заглавными буквами русского алфавита,
%% начиная с А, за исключением букв Ё, З, Й, О, Ч, Ь, Ы, Ъ.
%% Здесь также переделаны все нумерации русскими буквами.
\ifxetexorluatex
    \makeatletter
    \def\russian@Alph#1{\ifcase#1\or
       А\or Б\or В\or Г\or Д\or Е\or Ж\or
       И\or К\or Л\or М\or Н\or
       П\or Р\or С\or Т\or У\or Ф\or Х\or
       Ц\or Ш\or Щ\or Э\or Ю\or Я\else\xpg@ill@value{#1}{russian@Alph}\fi}
    \def\russian@alph#1{\ifcase#1\or
       а\or б\or в\or г\or д\or е\or ж\or
       и\or к\or л\or м\or н\or
       п\or р\or с\or т\or у\or ф\or х\or
       ц\or ш\or щ\or э\or ю\or я\else\xpg@ill@value{#1}{russian@alph}\fi}
    \def\cyr@Alph#1{\ifcase#1\or
        А\or Б\or В\or Г\or Д\or Е\or Ж\or
        И\or К\or Л\or М\or Н\or
        П\or Р\or С\or Т\or У\or Ф\or Х\or
        Ц\or Ш\or Щ\or Э\or Ю\or Я\else\xpg@ill@value{#1}{cyr@Alph}\fi}
    \def\cyr@alph#1{\ifcase#1\or
        а\or б\or в\or г\or д\or е\or ж\or
        и\or к\or л\or м\or н\or
        п\or р\or с\or т\or у\or ф\or х\or
        ц\or ш\or щ\or э\or ю\or я\else\xpg@ill@value{#1}{cyr@alph}\fi}
    \makeatother
\else
    \makeatletter
    \if@uni@ode
      \def\russian@Alph#1{\ifcase#1\or
        А\or Б\or В\or Г\or Д\or Е\or Ж\or
        И\or К\or Л\or М\or Н\or
        П\or Р\or С\or Т\or У\or Ф\or Х\or
        Ц\or Ш\or Щ\or Э\or Ю\or Я\else\@ctrerr\fi}
    \else
      \def\russian@Alph#1{\ifcase#1\or
        \CYRA\or\CYRB\or\CYRV\or\CYRG\or\CYRD\or\CYRE\or\CYRZH\or
        \CYRI\or\CYRK\or\CYRL\or\CYRM\or\CYRN\or
        \CYRP\or\CYRR\or\CYRS\or\CYRT\or\CYRU\or\CYRF\or\CYRH\or
        \CYRC\or\CYRSH\or\CYRSHCH\or\CYREREV\or\CYRYU\or
        \CYRYA\else\@ctrerr\fi}
    \fi
    \if@uni@ode
      \def\russian@alph#1{\ifcase#1\or
        а\or б\or в\or г\or д\or е\or ж\or
        и\or к\or л\or м\or н\or
        п\or р\or с\or т\or у\or ф\or х\or
        ц\or ш\or щ\or э\or ю\or я\else\@ctrerr\fi}
    \else
      \def\russian@alph#1{\ifcase#1\or
        \cyra\or\cyrb\or\cyrv\or\cyrg\or\cyrd\or\cyre\or\cyrzh\or
        \cyri\or\cyrk\or\cyrl\or\cyrm\or\cyrn\or
        \cyrp\or\cyrr\or\cyrs\or\cyrt\or\cyru\or\cyrf\or\cyrh\or
        \cyrc\or\cyrsh\or\cyrshch\or\cyrerev\or\cyryu\or
        \cyrya\else\@ctrerr\fi}
    \fi
    \makeatother
\fi

%%% Команды рецензирования %%%
\ifboolexpr{ (test {\ifnumequal{\value{draft}}{1}}) or (test {\ifnumequal{\value{showmarkup}}{1}})}{
        \newrobustcmd{\todo}[1]{\textcolor{red}{#1}}
        \newrobustcmd{\note}[2][]{\ifstrempty{#1}{#2}{\textcolor{#1}{#2}}}
        \newenvironment{commentbox}[1][]%
        {\ifstrempty{#1}{}{\color{#1}}}%
        {}
}{
        \newrobustcmd{\todo}[1]{}
        \newrobustcmd{\note}[2][]{}
        \excludecomment{commentbox}
}

           % Стили общие для диссертации и автореферата
%%% Переопределение именований, если иначе не сработает %%%
%\gappto\captionsrussian{
%    \renewcommand{\chaptername}{Глава}
%    \renewcommand{\appendixname}{Приложение} % (ГОСТ Р 7.0.11-2011, 5.7)
%}

%%% Изображения %%%
\graphicspath{{images/}{Dissertation/images/}}         % Пути к изображениям

%%% Интервалы %%%
%% По ГОСТ Р 7.0.11-2011, пункту 5.3.6 требуется полуторный интервал
%% Реализация средствами класса (на основе setspace) ближе к типографской классике.
%% И правит сразу и в таблицах (если со звёздочкой)
%\DoubleSpacing*     % Двойной интервал
\OnehalfSpacing*    % Полуторный интервал
%\setSpacing{1.42}   % Полуторный интервал, подобный Ворду (возможно, стоит включать вместе с предыдущей строкой)

%%% Макет страницы %%%
% Выставляем значения полей (ГОСТ 7.0.11-2011, 5.3.7)
\geometry{a4paper, top=2cm, bottom=2cm, left=2.5cm, right=1cm, nofoot, nomarginpar} %, heightrounded, showframe
\setlength{\topskip}{0pt}   %размер дополнительного верхнего поля
\setlength{\footskip}{12.3pt} % снимет warning, согласно https://tex.stackexchange.com/a/334346

%%% Выравнивание и переносы %%%
%% http://tex.stackexchange.com/questions/241343/what-is-the-meaning-of-fussy-sloppy-emergencystretch-tolerance-hbadness
%% http://www.latex-community.org/forum/viewtopic.php?p=70342#p70342
\tolerance 1414
\hbadness 1414
\emergencystretch 1.5em % В случае проблем регулировать в первую очередь
\hfuzz 0.3pt
\vfuzz \hfuzz
%\raggedbottom
%\sloppy                 % Избавляемся от переполнений
\clubpenalty=10000      % Запрещаем разрыв страницы после первой строки абзаца
\widowpenalty=10000     % Запрещаем разрыв страницы после последней строки абзаца
\brokenpenalty=4991     % Ограничение на разрыв страницы, если строка заканчивается переносом

%%% Блок управления параметрами для выравнивания заголовков в тексте %%%
\newlength{\otstuplen}
\setlength{\otstuplen}{\theotstup\parindent}
\ifnumequal{\value{headingalign}}{0}{% выравнивание заголовков в тексте
    \newcommand{\hdngalign}{\centering}                % по центру
    \newcommand{\hdngaligni}{}% по центру
    \setlength{\otstuplen}{0pt}
}{%
    \newcommand{\hdngalign}{}                 % по левому краю
    \newcommand{\hdngaligni}{\hspace{\otstuplen}}      % по левому краю
} % В обоих случаях вроде бы без переноса, как и надо (ГОСТ Р 7.0.11-2011, 5.3.5)

%%% Оглавление %%%
\renewcommand{\cftchapterdotsep}{\cftdotsep}                % отбивка точками до номера страницы начала главы/раздела

%% Переносить слова в заголовке не допускается (ГОСТ Р 7.0.11-2011, 5.3.5). Заголовки в оглавлении должны точно повторять заголовки в тексте (ГОСТ Р 7.0.11-2011, 5.2.3). Прямого указания на запрет переносов в оглавлении нет, но по той же логике невнесения искажений в смысл, лучше в оглавлении не переносить:
\setrmarg{2.55em plus1fil}                             %To have the (sectional) titles in the ToC, etc., typeset ragged right with no hyphenation
\renewcommand{\cftchapterpagefont}{\normalfont}        % нежирные номера страниц у глав в оглавлении
\renewcommand{\cftchapterleader}{\cftdotfill{\cftchapterdotsep}}% нежирные точки до номеров страниц у глав в оглавлении
%\renewcommand{\cftchapterfont}{}                       % нежирные названия глав в оглавлении

\ifnumgreater{\value{headingdelim}}{0}{%
    \renewcommand\cftchapteraftersnum{.\space}       % добавляет точку с пробелом после номера раздела в оглавлении
}{}
\ifnumgreater{\value{headingdelim}}{1}{%
    \renewcommand\cftsectionaftersnum{.\space}       % добавляет точку с пробелом после номера подраздела в оглавлении
    \renewcommand\cftsubsectionaftersnum{.\space}    % добавляет точку с пробелом после номера подподраздела в оглавлении
    \renewcommand\cftsubsubsectionaftersnum{.\space} % добавляет точку с пробелом после номера подподподраздела в оглавлении
    \AfterEndPreamble{% без этого polyglossia сама всё переопределяет
        \setsecnumformat{\csname the#1\endcsname.\space}
    }
}{%
    \AfterEndPreamble{% без этого polyglossia сама всё переопределяет
        \setsecnumformat{\csname the#1\endcsname\quad}
    }
}

\renewcommand*{\cftappendixname}{\appendixname\space} % Слово Приложение в оглавлении

%%% Колонтитулы %%%
% Порядковый номер страницы печатают на середине верхнего поля страницы (ГОСТ Р 7.0.11-2011, 5.3.8)
\makeevenhead{plain}{}{}{}
\makeoddhead{plain}{}{}{}
\makeevenfoot{plain}{}{\thepage}{}
\makeoddfoot{plain}{}{\thepage}{}
\pagestyle{plain}

%%% добавить Стр. над номерами страниц в оглавлении
%%% http://tex.stackexchange.com/a/306950
\newif\ifendTOC

\newcommand*{\tocheader}{
\ifnumequal{\value{pgnum}}{1}{%
    \ifendTOC\else\hbox to \linewidth%
      {\noindent{}~\hfill{Стр.}}\par%
      \ifnumless{\value{page}}{3}{}{%
        \vspace{0.5\onelineskip}
      }
      \afterpage{\tocheader}
    \fi%
}{}%
}%

%%% Оформление заголовков глав, разделов, подразделов %%%
%% Работа должна быть выполнена ... размером шрифта 12-14 пунктов (ГОСТ Р 7.0.11-2011, 5.3.8). То есть не должно быть надписей шрифтом более 14. Так и поставим.
%% Эти установки будут давать одинаковый результат независимо от выбора базовым шрифтом 12 пт или 14 пт
\newcommand{\basegostsectionfont}{\fontsize{14pt}{16pt}\selectfont\bfseries}

\makechapterstyle{thesisgost}{%
    \chapterstyle{default}
    \setlength{\beforechapskip}{0pt}
    \setlength{\midchapskip}{0pt}
    \setlength{\afterchapskip}{\theintvl\curtextsize}
    \renewcommand*{\chapnamefont}{\basegostsectionfont}
    \renewcommand*{\chapnumfont}{\basegostsectionfont}
    \renewcommand*{\chaptitlefont}{\basegostsectionfont}
    \renewcommand*{\chapterheadstart}{}
    \ifnumgreater{\value{headingdelim}}{0}{%
        \renewcommand*{\afterchapternum}{.\space}   % добавляет точку с пробелом после номера раздела
    }{%
        \renewcommand*{\afterchapternum}{\quad}     % добавляет \quad после номера раздела
    }
    \renewcommand*{\printchapternum}{\hdngaligni\hdngalign\chapnumfont \thechapter}
    \renewcommand*{\printchaptername}{}
    \renewcommand*{\printchapternonum}{\hdngaligni\hdngalign}
}

\makeatletter
\makechapterstyle{thesisgostchapname}{%
    \chapterstyle{thesisgost}
    \renewcommand*{\printchapternum}{\chapnumfont \thechapter}
    \renewcommand*{\printchaptername}{\hdngaligni\hdngalign\chapnamefont \@chapapp} %
}
\makeatother

\chapterstyle{thesisgost}

\setsecheadstyle{\basegostsectionfont\hdngalign}
\setsecindent{\otstuplen}

\setsubsecheadstyle{\basegostsectionfont\hdngalign}
\setsubsecindent{\otstuplen}

\setsubsubsecheadstyle{\basegostsectionfont\hdngalign}
\setsubsubsecindent{\otstuplen}

\sethangfrom{\noindent #1} %все заголовки подразделов центрируются с учетом номера, как block

\ifnumequal{\value{chapstyle}}{1}{%
    \chapterstyle{thesisgostchapname}
    \renewcommand*{\cftchaptername}{\chaptername\space} % будет вписано слово Глава перед каждым номером раздела в оглавлении
}{}%

%%% Интервалы между заголовками
\setbeforesecskip{\theintvl\curtextsize}% Заголовки отделяют от текста сверху и снизу тремя интервалами (ГОСТ Р 7.0.11-2011, 5.3.5).
\setaftersecskip{\theintvl\curtextsize}
\setbeforesubsecskip{\theintvl\curtextsize}
\setaftersubsecskip{\theintvl\curtextsize}
\setbeforesubsubsecskip{\theintvl\curtextsize}
\setaftersubsubsecskip{\theintvl\curtextsize}

%%% Вертикальные интервалы глав (\chapter) в оглавлении как и у заголовков
% раскомментировать следующие 2
% \setlength{\cftbeforechapterskip}{0pt plus 0pt}   % ИЛИ эти 2 строки из учебника
% \renewcommand*{\insertchapterspace}{}
% или эту  
% \renewcommand*{\cftbeforechapterskip}{0em}       


%%% Блок дополнительного управления размерами заголовков
\ifnumequal{\value{headingsize}}{1}{% Пропорциональные заголовки и базовый шрифт 14 пт
    \renewcommand{\basegostsectionfont}{\large\bfseries}
    \renewcommand*{\chapnamefont}{\Large\bfseries}
    \renewcommand*{\chapnumfont}{\Large\bfseries}
    \renewcommand*{\chaptitlefont}{\Large\bfseries}
}{}

%%% Счётчики %%%

%% Упрощённые настройки шаблона диссертации: нумерация формул, таблиц, рисунков
\ifnumequal{\value{contnumeq}}{1}{%
    \counterwithout{equation}{chapter} % Убираем связанность номера формулы с номером главы/раздела
}{}
\ifnumequal{\value{contnumfig}}{1}{%
    \counterwithout{figure}{chapter}   % Убираем связанность номера рисунка с номером главы/раздела
}{}
\ifnumequal{\value{contnumtab}}{1}{%
    \counterwithout{table}{chapter}    % Убираем связанность номера таблицы с номером главы/раздела
}{}


%%http://www.linux.org.ru/forum/general/6993203#comment-6994589 (используется totcount)
\makeatletter
\def\formbytotal#1#2#3#4#5{%
    \newcount\@c
    \@c\totvalue{#1}\relax
    \newcount\@last
    \newcount\@pnul
    \@last\@c\relax
    \divide\@last 10
    \@pnul\@last\relax
    \divide\@pnul 10
    \multiply\@pnul-10
    \advance\@pnul\@last
    \multiply\@last-10
    \advance\@last\@c
    \total{#1}~#2%
    \ifnum\@pnul=1#5\else%
    \ifcase\@last#5\or#3\or#4\or#4\or#4\else#5\fi
    \fi
}
\makeatother

\AtBeginDocument{
%% регистрируем счётчики в системе totcounter
    \regtotcounter{totalcount@figure}
    \regtotcounter{totalcount@table}       % Если иным способом поставить в преамбуле то ошибка в числе таблиц
    \regtotcounter{TotPages}               % Если иным способом поставить в преамбуле то ошибка в числе страниц
}
  % Стили для диссертации
\input{Dissertation/userstyles} % Стили для специфических пользовательских задач

%%% Библиография. Выбор движка для реализации %%%
% Здесь только проверка установленного ключа. Сама настройка выбора движка
% размещена в common/setup.tex
\ifnumequal{\value{bibliosel}}{0}{%
    %%% Реализация библиографии встроенными средствами посредством движка bibtex8 %%%

%%% Пакеты %%%
\usepackage{cite}                                   % Красивые ссылки на литературу


%%% Стили %%%
\bibliographystyle{BibTeX-Styles/utf8gost71u}    % Оформляем библиографию по ГОСТ 7.1 (ГОСТ Р 7.0.11-2011, 5.6.7)

\makeatletter
\renewcommand{\@biblabel}[1]{#1.}   % Заменяем библиографию с квадратных скобок на точку
\makeatother
%% Управление отступами между записями
%% требует etoolbox
%% http://tex.stackexchange.com/a/105642
%\patchcmd\thebibliography
% {\labelsep}
% {\labelsep\itemsep=5pt\parsep=0pt\relax}
% {}
% {\typeout{Couldn't patch the command}}

%%% Список литературы с красной строки (без висячего отступа) %%%
%\patchcmd{\thebibliography} %может потребовать включения пакета etoolbox
%  {\advance\leftmargin\labelsep}
%  {\leftmargin=0pt%
%   \setlength{\labelsep}{\widthof{\ }}% Управляет длиной отступа после точки
%   \itemindent=\parindent%
%   \addtolength{\itemindent}{\labelwidth}% Сдвигаем правее на величину номера с точкой
%   \advance\itemindent\labelsep%
%  }
%  {}{}

%%% Цитирование %%%
\renewcommand\citepunct{;\penalty\citepunctpenalty%
    \hskip.13emplus.1emminus.1em\relax}                % Разделение ; при перечислении ссылок (ГОСТ Р 7.0.5-2008)

\newcommand*{\autocite}[1]{}  % Чтобы примеры цитирования, рассчитанные на biblatex, не вызывали ошибок при компиляции в bibtex

%%% Создание команд для вывода списка литературы %%%
\newcommand*{\insertbibliofull}{
\bibliography{biblio/external,biblio/author}         % Подключаем BibTeX-базы % После запятых не должно быть лишних пробелов — он "думает", что это тоже имя пути
}

\newcommand*{\insertbiblioauthor}{
\bibliography{biblio/author}         % Подключаем BibTeX-базы % После запятых не должно быть лишних пробелов — он "думает", что это тоже имя пути
}

\newcommand*{\insertbiblioexternal}{
\bibliography{biblio/external}         % Подключаем BibTeX-базы
}


%% Счётчик использованных ссылок на литературу, обрабатывающий с учётом неоднократных ссылок
%% Требуется дважды компилировать, поскольку ему нужно считать актуальный внешний файл со списком литературы
\newtotcounter{citenum}
\def\oldcite{}
\let\oldcite=\bibcite
\def\bibcite{\stepcounter{citenum}\oldcite}
   % Встроенная реализация с загрузкой файла через движок bibtex8
}{
    %%% Реализация библиографии пакетами biblatex и biblatex-gost с использованием движка biber %%%

\usepackage{csquotes} % biblatex рекомендует его подключать. Пакет для оформления сложных блоков цитирования.
%%% Загрузка пакета с основными настройками %%%
\makeatletter
\ifnumequal{\value{draft}}{0}{% Чистовик
\usepackage[%
backend=biber,% движок
bibencoding=utf8,% кодировка bib файла
sorting=nty,% настройка сортировки списка литературы
style=gost-numeric,% стиль цитирования и библиографии (по ГОСТ)
language=autobib,% получение языка из babel/polyglossia, default: autobib % если ставить autocite или auto, то цитаты в тексте с указанием страницы, получат указание страницы на языке оригинала
autolang=other,% многоязычная библиография
clearlang=true,% внутренний сброс поля language, если он совпадает с языком из babel/polyglossia
defernumbers=true,% нумерация проставляется после двух компиляций, зато позволяет выцеплять библиографию по ключевым словам и нумеровать не из большего списка
sortcites=false,% сортировать номера затекстовых ссылок при цитировании (если в квадратных скобках несколько ссылок, то отображаться будут отсортированно, а не абы как)
doi=false,% Показывать или нет ссылки на DOI
isbn=false,% Показывать или нет ISBN, ISSN, ISRN
]{biblatex}[2016/09/17]
%\ltx@iffilelater{biblatex-gost.def}{2017/05/03}%
%{\toggletrue{bbx:gostbibliography}%
%\renewcommand*{\revsdnamepunct}{\addcomma}}{}
}{%Черновик
\usepackage[%
backend=biber,% движок
bibencoding=utf8,% кодировка bib файла
sorting=nty,% настройка сортировки списка литературы
% defernumbers=true, % откомментируйте, если требуется правильная нумерация ссылок на литературу в режиме черновика. Замедляет сборку
]{biblatex}[2016/09/17]%
}
\makeatother

\ifxetexorluatex
\else
% Исправление случая неподдержки знака номера в pdflatex
    \DefineBibliographyStrings{russian}{number={\textnumero}}
\fi

\ifsynopsis
\ifnumgreater{\value{usefootcite}}{0}{
    \ExecuteBibliographyOptions{autocite=footnote}
    \newbibmacro*{cite:full}{%
        \printtext[bibhypertarget]{%
            \usedriver{%
                \DeclareNameAlias{sortname}{default}%
            }{%
                \thefield{entrytype}%
            }%
        }%
        \usebibmacro{shorthandintro}%
    }
    \DeclareCiteCommand{\smartcite}[\mkbibfootnote]{%
        \usebibmacro{prenote}%
    }{%
        \usebibmacro{citeindex}%
        \usebibmacro{cite:full}%
    }{%
        \multicitedelim%
    }{%
        \usebibmacro{postnote}%
    }
}{}
\fi

%%% Подключение файлов bib %%%
\addbibresource[label=bl-external]{biblio/external.bib}
\addbibresource[label=bl-author]{biblio/author.bib}

%http://tex.stackexchange.com/a/141831/79756
%There is a way to automatically map the language field to the langid field. The following lines in the preamble should be enough to do that.
%This command will copy the language field into the langid field and will then delete the contents of the language field. The language field will only be deleted if it was successfully copied into the langid field.
\DeclareSourcemap{ %модификация bib файла перед тем, как им займётся biblatex
    \maps{
        \map{% перекидываем значения полей language в поля langid, которыми пользуется biblatex
            \step[fieldsource=language, fieldset=langid, origfieldval, final]
            \step[fieldset=language, null]
        }
        \map{% перекидываем значения полей numpages в поля pagetotal, которыми пользуется biblatex
            \step[fieldsource=numpages, fieldset=pagetotal, origfieldval, final]
            \step[fieldset=numpages, null]
        }
        \map{% перекидываем значения полей pagestotal в поля pagetotal, которыми пользуется biblatex
            \step[fieldsource=pagestotal, fieldset=pagetotal, origfieldval, final]
            \step[fieldset=pagestotal, null]
        }
        \map[overwrite]{% перекидываем значения полей shortjournal, если они есть, в поля journal, которыми пользуется biblatex
            \step[fieldsource=shortjournal, final]
            \step[fieldset=journal, origfieldval]
            \step[fieldset=shortjournal, null]
        }
        \map[overwrite]{% перекидываем значения полей shortbooktitle, если они есть, в поля booktitle, которыми пользуется biblatex
            \step[fieldsource=shortbooktitle, final]
            \step[fieldset=booktitle, origfieldval]
            \step[fieldset=shortbooktitle, null]
        }
        \map{% если в поле medium написано "Электронный ресурс", то устанавливаем поле media, которым пользуется biblatex, в значение eresource.
            \step[fieldsource=medium,
            match=\regexp{Электронный\s+ресурс},
            final]
            \step[fieldset=media, fieldvalue=eresource]
            \step[fieldset=medium, null]
        }
        \map{% использование media=text по умолчанию
            \step[fieldset=media, fieldvalue=text]
        }
        \map[overwrite]{% стираем значения всех полей issn
            \step[fieldset=issn, null]
        }
        \map[overwrite]{% стираем значения всех полей abstract, поскольку ими не пользуемся, а там бывают "неприятные" латеху символы
            \step[fieldsource=abstract]
            \step[fieldset=abstract,null]
        }
        \map[overwrite]{ % переделка формата записи даты
            \step[fieldsource=urldate,
            match=\regexp{([0-9]{2})\.([0-9]{2})\.([0-9]{4})},
            replace={$3-$2-$1$4}, % $4 вставлен исключительно ради нормальной работы программ подсветки синтаксиса, которые некорректно обрабатывают $ в таких конструкциях
            final]
        }
        \map[overwrite]{ % стираем ключевые слова
            \step[fieldsource=keywords]
            \step[fieldset=keywords,null]
        }
        % реализация foreach различается для biblatex v3.12 и v3.13.
        % Для версии v3.13 эта конструкция заменяет последующие 5 структур map
        % \map[overwrite,foreach={authorvak,authorscopus,authorwos,authorconf,authorother}]{ % записываем информацию о типе публикации в ключевые слова
        %     \step[fieldsource=$MAPLOOP,final=true]
        %     \step[fieldset=keywords,fieldvalue={,biblio$MAPLOOP},append=true]
        % }
        \map[overwrite]{ % записываем информацию о типе публикации в ключевые слова
            \step[fieldsource=authorvak,final=true]
            \step[fieldset=keywords,fieldvalue={,biblioauthorvak},append=true]
        }
        \map[overwrite]{ % записываем информацию о типе публикации в ключевые слова
            \step[fieldsource=authorscopus,final=true]
            \step[fieldset=keywords,fieldvalue={,biblioauthorscopus},append=true]
        }
        \map[overwrite]{ % записываем информацию о типе публикации в ключевые слова
            \step[fieldsource=authorwos,final=true]
            \step[fieldset=keywords,fieldvalue={,biblioauthorwos},append=true]
        }
        \map[overwrite]{ % записываем информацию о типе публикации в ключевые слова
            \step[fieldsource=authorconf,final=true]
            \step[fieldset=keywords,fieldvalue={,biblioauthorconf},append=true]
        }
        \map[overwrite]{ % записываем информацию о типе публикации в ключевые слова
            \step[fieldsource=authorother,final=true]
            \step[fieldset=keywords,fieldvalue={,biblioauthorother},append=true]
        }
        \map[overwrite]{ % добавляем ключевые слова, чтобы различать источники
            \perdatasource{biblio/external.bib}
            \step[fieldset=keywords, fieldvalue={,biblioexternal},append=true]
        }
        \map[overwrite]{ % добавляем ключевые слова, чтобы различать источники
            \perdatasource{biblio/author.bib}
            \step[fieldset=keywords, fieldvalue={,biblioauthor},append=true]
        }
        \map[overwrite]{ % добавляем ключевые слова, чтобы различать источники
            \step[fieldset=keywords, fieldvalue={,bibliofull},append=true]
        }
%        \map[overwrite]{% стираем значения всех полей series
%            \step[fieldset=series, null]
%        }
        \map[overwrite]{% перекидываем значения полей howpublished в поля organization для типа online
            \step[typesource=online, typetarget=online, final]
            \step[fieldsource=howpublished, fieldset=organization, origfieldval]
            \step[fieldset=howpublished, null]
        }
        % Так отключаем [Электронный ресурс]
       \map[overwrite]{% стираем значения всех полей media=eresource
           \step[fieldsource=media,
           match={eresource},
           final]
           \step[fieldset=media, null]
       }
    }
}

\ifsynopsis
\else
\DeclareSourcemap{ %модификация bib файла перед тем, как им займётся biblatex
    \maps{
        \map[overwrite]{% стираем значения всех полей addendum
            \perdatasource{biblio/author.bib}
            \step[fieldset=addendum, null] %чтобы избавиться от информации об объёме авторских статей, в отличие от автореферата
        }
    }
}
\fi

\defbibfilter{vakscopuswos}{%
    keyword=biblioauthorvak or keyword=biblioauthorscopus or keyword=biblioauthorwos
}

\defbibfilter{scopuswos}{%
    keyword=biblioauthorscopus or keyword=biblioauthorwos
}

%%% Убираем неразрывные пробелы перед двоеточием и точкой с запятой %%%
%\makeatletter
%\ifnumequal{\value{draft}}{0}{% Чистовик
%    \renewcommand*{\addcolondelim}{%
%      \begingroup%
%      \def\abx@colon{%
%        \ifdim\lastkern>\z@\unkern\fi%
%        \abx@puncthook{:}\space}%
%      \addcolon%
%      \endgroup}
%
%    \renewcommand*{\addsemicolondelim}{%
%      \begingroup%
%      \def\abx@semicolon{%
%        \ifdim\lastkern>\z@\unkern\fi%
%        \abx@puncthook{;}\space}%
%      \addsemicolon%
%      \endgroup}
%}{}
%\makeatother

%%% Правка записей типа thesis, чтобы дважды не писался автор
%\ifnumequal{\value{draft}}{0}{% Чистовик
%\DeclareBibliographyDriver{thesis}{%
%  \usebibmacro{bibindex}%
%  \usebibmacro{begentry}%
%  \usebibmacro{heading}%
%  \newunit
%  \usebibmacro{author}%
%  \setunit*{\labelnamepunct}%
%  \usebibmacro{thesistitle}%
%  \setunit{\respdelim}%
%  %\printnames[last-first:full]{author}%Вот эту строчку нужно убрать, чтобы автор диссертации не дублировался
%  \newunit\newblock
%  \printlist[semicolondelim]{specdata}%
%  \newunit
%  \usebibmacro{institution+location+date}%
%  \newunit\newblock
%  \usebibmacro{chapter+pages}%
%  \newunit
%  \printfield{pagetotal}%
%  \newunit\newblock
%  \usebibmacro{doi+eprint+url+note}%
%  \newunit\newblock
%  \usebibmacro{addendum+pubstate}%
%  \setunit{\bibpagerefpunct}\newblock
%  \usebibmacro{pageref}%
%  \newunit\newblock
%  \usebibmacro{related:init}%
%  \usebibmacro{related}%
%  \usebibmacro{finentry}}
%}{}

%\newbibmacro{string+doi}[1]{% новая макрокоманда на простановку ссылки на doi
%    \iffieldundef{doi}{#1}{\href{http://dx.doi.org/\thefield{doi}}{#1}}}

%\ifnumequal{\value{draft}}{0}{% Чистовик
%\renewcommand*{\mkgostheading}[1]{\usebibmacro{string+doi}{#1}} % ссылка на doi с авторов. стоящих впереди записи
%\renewcommand*{\mkgostheading}[1]{#1} % только лишь убираем курсив с авторов
%}{}
%\DeclareFieldFormat{title}{\usebibmacro{string+doi}{#1}} % ссылка на doi с названия работы
%\DeclareFieldFormat{journaltitle}{\usebibmacro{string+doi}{#1}} % ссылка на doi с названия журнала
%%% Тире как разделитель в библиографии традиционной руской длины:
\renewcommand*{\newblockpunct}{\addperiod\addnbspace\cyrdash\space\bibsentence}
%%% Убрать тире из разделителей элементов в библиографии:
%\renewcommand*{\newblockpunct}{%
%    \addperiod\space\bibsentence}%block punct.,\bibsentence is for vol,etc.

%%% Возвращаем запись «Режим доступа» %%%
%\DefineBibliographyStrings{english}{%
%    urlfrom = {Mode of access}
%}
%\DeclareFieldFormat{url}{\bibstring{urlfrom}\addcolon\space\url{#1}}

%%% В списке литературы обозначение одной буквой диапазона страниц англоязычного источника %%%
\DefineBibliographyStrings{english}{%
    pages = {p\adddot} %заглавность буквы затем по месту определяется работой самого biblatex
}

%%% В ссылке на источник в основном тексте с указанием конкретной страницы обозначение одной большой буквой %%%
%\DefineBibliographyStrings{russian}{%
%    page = {C\adddot}
%}

%%% Исправление длины тире в диапазонах %%%
% \cyrdash --- тире «русской» длины, \textendash --- en-dash
\DefineBibliographyExtras{russian}{%
  \protected\def\bibrangedash{%
    \cyrdash\penalty\value{abbrvpenalty}}% almost unbreakable dash
  \protected\def\bibdaterangesep{\bibrangedash}%тире для дат
}
\DefineBibliographyExtras{english}{%
  \protected\def\bibrangedash{%
    \cyrdash\penalty\value{abbrvpenalty}}% almost unbreakable dash
  \protected\def\bibdaterangesep{\bibrangedash}%тире для дат
}

%Set higher penalty for breaking in number, dates and pages ranges
\setcounter{abbrvpenalty}{10000} % default is \hyphenpenalty which is 12

%Set higher penalty for breaking in names
\setcounter{highnamepenalty}{10000} % If you prefer the traditional BibTeX behavior (no linebreaks at highnamepenalty breakpoints), set it to ‘infinite’ (10 000 or higher).
\setcounter{lownamepenalty}{10000}

%%% Set low penalties for breaks at uppercase letters and lowercase letters
%\setcounter{biburllcpenalty}{500} %управляет разрывами ссылок после маленьких букв RTFM biburllcpenalty
%\setcounter{biburlucpenalty}{3000} %управляет разрывами ссылок после больших букв, RTFM biburlucpenalty

%%% Список литературы с красной строки (без висячего отступа) %%%
%\defbibenvironment{bibliography} % переопределяем окружение библиографии из gost-numeric.bbx пакета biblatex-gost
%  {\list
%     {\printtext[labelnumberwidth]{%
%       \printfield{prefixnumber}%
%       \printfield{labelnumber}}}
%     {%
%      \setlength{\labelwidth}{\labelnumberwidth}%
%      \setlength{\leftmargin}{0pt}% default is \labelwidth
%      \setlength{\labelsep}{\widthof{\ }}% Управляет длиной отступа после точки % default is \biblabelsep
%      \setlength{\itemsep}{\bibitemsep}% Управление дополнительным вертикальным разрывом между записями. \bibitemsep по умолчанию соответствует \itemsep списков в документе.
%      \setlength{\itemindent}{\bibhang}% Пользуемся тем, что \bibhang по умолчанию принимает значение \parindent (абзацного отступа), который переназначен в styles.tex
%      \addtolength{\itemindent}{\labelwidth}% Сдвигаем правее на величину номера с точкой
%      \addtolength{\itemindent}{\labelsep}% Сдвигаем ещё правее на отступ после точки
%      \setlength{\parsep}{\bibparsep}%
%     }%
%      \renewcommand*{\makelabel}[1]{\hss##1}%
%  }
%  {\endlist}
%  {\item}

%%% Макросы автоматического подсчёта количества авторских публикаций.
% Печатают невидимую (пустую) библиографию, считая количество источников.
% http://tex.stackexchange.com/a/66851/79756
%
\makeatletter
        \newtotcounter{citenum}
        \defbibenvironment{counter}
            {\setcounter{citenum}{0}\renewcommand{\blx@driver}[1]{}} % begin code: убирает весь выводимый текст
            {} % end code
            {\stepcounter{citenum}} % item code: cчитает "печатаемые в библиографию" источники

        \newtotcounter{citeauthorvak}
        \defbibenvironment{countauthorvak}
            {\setcounter{citeauthorvak}{0}\renewcommand{\blx@driver}[1]{}}
            {}
            {\stepcounter{citeauthorvak}}

        \newtotcounter{citeauthorscopus}
        \defbibenvironment{countauthorscopus}
                {\setcounter{citeauthorscopus}{0}\renewcommand{\blx@driver}[1]{}}
                {}
                {\stepcounter{citeauthorscopus}}

        \newtotcounter{citeauthorwos}
        \defbibenvironment{countauthorwos}
                {\setcounter{citeauthorwos}{0}\renewcommand{\blx@driver}[1]{}}
                {}
                {\stepcounter{citeauthorwos}}

        \newtotcounter{citeauthorother}
        \defbibenvironment{countauthorother}
                {\setcounter{citeauthorother}{0}\renewcommand{\blx@driver}[1]{}}
                {}
                {\stepcounter{citeauthorother}}

        \newtotcounter{citeauthorconf}
        \defbibenvironment{countauthorconf}
                {\setcounter{citeauthorconf}{0}\renewcommand{\blx@driver}[1]{}}
                {}
                {\stepcounter{citeauthorconf}}

        \newtotcounter{citeauthor}
        \defbibenvironment{countauthor}
                {\setcounter{citeauthor}{0}\renewcommand{\blx@driver}[1]{}}
                {}
                {\stepcounter{citeauthor}}

        \newtotcounter{citeauthorvakscopuswos}
        \defbibenvironment{countauthorvakscopuswos}
                {\setcounter{citeauthorvakscopuswos}{0}\renewcommand{\blx@driver}[1]{}}
                {}
                {\stepcounter{citeauthorvakscopuswos}}

        \newtotcounter{citeauthorscopuswos}
        \defbibenvironment{countauthorscopuswos}
                {\setcounter{citeauthorscopuswos}{0}\renewcommand{\blx@driver}[1]{}}
                {}
                {\stepcounter{citeauthorscopuswos}}

        \newtotcounter{citeexternal}
        \defbibenvironment{countexternal}
                {\setcounter{citeexternal}{0}\renewcommand{\blx@driver}[1]{}}
                {}
                {\stepcounter{citeexternal}}
\makeatother

\defbibheading{nobibheading}{} % пустой заголовок, для подсчёта публикаций с помощью невидимой библиографии
\defbibheading{pubgroup}{\section*{#1}} % обычный стиль, заголовок-секция
\defbibheading{pubsubgroup}{\noindent\textbf{#1}} % для подразделов "по типу источника"

%%%Сортировка списка литературы Русский-Английский (предварительно удалить dissertation.bbl) (начало)
%%%Источник: https://github.com/odomanov/biblatex-gost/wiki/%D0%9A%D0%B0%D0%BA-%D1%81%D0%B4%D0%B5%D0%BB%D0%B0%D1%82%D1%8C,-%D1%87%D1%82%D0%BE%D0%B1%D1%8B-%D1%80%D1%83%D1%81%D1%81%D0%BA%D0%BE%D1%8F%D0%B7%D1%8B%D1%87%D0%BD%D1%8B%D0%B5-%D0%B8%D1%81%D1%82%D0%BE%D1%87%D0%BD%D0%B8%D0%BA%D0%B8-%D0%BF%D1%80%D0%B5%D0%B4%D1%88%D0%B5%D1%81%D1%82%D0%B2%D0%BE%D0%B2%D0%B0%D0%BB%D0%B8-%D0%BE%D1%81%D1%82%D0%B0%D0%BB%D1%8C%D0%BD%D1%8B%D0%BC
\DeclareSourcemap{
	\maps[datatype=bibtex]{
		\map{
			\step[fieldset=langid, fieldvalue={tempruorder}]
		}
		\map[overwrite]{
			\step[fieldsource=langid, match=russian, final]
			\step[fieldsource=presort, 
			match=\regexp{(.+)}, 
			replace=\regexp{aa$1}]
		}
		\map{
			\step[fieldsource=langid, match=russian, final]
			\step[fieldset=presort, fieldvalue={az}]
		}
		\map[overwrite]{
			\step[fieldsource=langid, notmatch=russian, final]
			\step[fieldsource=presort, 
			match=\regexp{(.+)}, 
			replace=\regexp{za$1}]
		}
		\map{
			\step[fieldsource=langid, notmatch=russian, final]
			\step[fieldset=presort, fieldvalue={zz}]
		}
		\map{
			\step[fieldsource=langid, match={tempruorder}, final]
			\step[fieldset=langid, null]
		}
	}
}
%Сортировка списка литературы (конец)

%%% Создание команд для вывода списка литературы %%%
\newcommand*{\insertbibliofull}{
    \printbibliography[keyword=bibliofull,section=0,title=\bibtitlefull]
    \ifnumequal{\value{draft}}{0}{
      \printbibliography[heading=nobibheading,env=counter,keyword=bibliofull,section=0]
    }{}
}
\newcommand*{\insertbiblioauthor}{
    \printbibliography[heading=pubgroup, section=0, keyword=biblioauthor, title=\bibtitleauthor]
}
\newcommand*{\insertbiblioauthorimportant}{
    \printbibliography[heading=pubgroup, section=2, keyword=biblioauthor, title=\bibtitleauthorimportant]
}

% Вариант вывода печатных работ автора, с группировкой по типу источника.
% Порядок команд `\printbibliography` должен соответствовать порядку в файле common/characteristic.tex
\newcommand*{\insertbiblioauthorgrouped}{
    \section*{\bibtitleauthor}
    \ifsynopsis
    \printbibliography[heading=pubsubgroup, section=0, keyword=biblioauthorvak,    title=\bibtitleauthorvak,resetnumbers=true] % Работы автора из списка ВАК (сброс нумерации)
    \else
    \printbibliography[heading=pubsubgroup, section=0, keyword=biblioauthorvak,    title=\bibtitleauthorvak,resetnumbers=false] % Работы автора из списка ВАК (сквозная нумерация)
    \fi
    \printbibliography[heading=pubsubgroup, section=0, keyword=biblioauthorwos,    title=\bibtitleauthorwos,resetnumbers=false]% Работы автора, индексируемые Web of Science
    \printbibliography[heading=pubsubgroup, section=0, keyword=biblioauthorscopus, title=\bibtitleauthorscopus,resetnumbers=false]% Работы автора, индексируемые Scopus
    \printbibliography[heading=pubsubgroup, section=0, keyword=biblioauthorconf,   title=\bibtitleauthorconf,resetnumbers=false]% Тезисы конференций
    \printbibliography[heading=pubsubgroup, section=0, keyword=biblioauthorother,  title=\bibtitleauthorother,resetnumbers=false]% Прочие работы автора
}

\newcommand*{\insertbiblioexternal}{
    \printbibliography[heading=pubgroup,    section=0, keyword=biblioexternal,     title=\bibtitlefull]
}
     % Реализация пакетом biblatex через движок biber
}

% Вывести информацию о выбранных опциях в лог сборки
\typeout{Selected options:}
\typeout{Draft mode: \arabic{draft}}
\typeout{Font: \arabic{fontfamily}}
\typeout{AltFont: \arabic{usealtfont}}
\typeout{Bibliography backend: \arabic{bibliosel}}
\typeout{Precompile images: \arabic{imgprecompile}}
% Вывести информацию о версиях используемых библиотек в лог сборки
\listfiles

%%% Управление компиляцией отдельных частей диссертации %%%
% Необходимо сначала иметь полностью скомпилированный документ, чтобы все
% промежуточные файлы были в наличии
% Затем, для вывода отдельных частей можно воспользоваться командой \includeonly
% Ниже примеры использования команды:
%
%\includeonly{Dissertation/chapter1}
%\includeonly{Dissertation/contents,Dissertation/appendix,Dissertation/conclusion}
%
% Если все команды закомментированы, то документ будет выведен в PDF файл полностью

\begin{document}

\input{common/renames}                 % Переопределение именований

%%% Структура диссертации (ГОСТ Р 7.0.11-2011, 4)
% Титульный лист (ГОСТ Р 7.0.11-2001, 5.1)
\thispagestyle{empty}
\begin{center}
\thesisOrganization
\end{center}
%
\vspace{0pt plus4fill} %число перед fill = кратность относительно некоторого расстояния fill, кусками которого заполнены пустые места
\IfFileExists{images/logo.pdf}{
  \begin{minipage}[b]{0.5\linewidth}
    \begin{flushleft}
      \includegraphics[height=3.5cm]{logo}
    \end{flushleft}
  \end{minipage}%
  \begin{minipage}[b]{0.5\linewidth}
    \begin{flushright}
      На правах рукописи\\
%      \textsl {УДК \thesisUdk}
    \end{flushright}
  \end{minipage}
}{
\begin{flushright}
На правах рукописи

%\textsl {УДК \thesisUdk}
\end{flushright}
}
%
\vspace{0pt plus6fill} %число перед fill = кратность относительно некоторого расстояния fill, кусками которого заполнены пустые места
\begin{center}
{\large \thesisAuthor}
\end{center}
%
\vspace{0pt plus1fill} %число перед fill = кратность относительно некоторого расстояния fill, кусками которого заполнены пустые места
\begin{center}
\textbf {\large %\MakeUppercase
\thesisTitle}

\vspace{0pt plus2fill} %число перед fill = кратность относительно некоторого расстояния fill, кусками которого заполнены пустые места
{%\small
Специальность \thesisSpecialtyNumber\ "---

<<\thesisSpecialtyTitle>>
}

\ifdefined\thesisSpecialtyTwoNumber
{%\small
Специальность \thesisSpecialtyTwoNumber\ "---

<<\thesisSpecialtyTwoTitle>>
}
\fi

\vspace{0pt plus2fill} %число перед fill = кратность относительно некоторого расстояния fill, кусками которого заполнены пустые места


\thesisDegree
\end{center}
%
\vspace{0pt plus4fill} %число перед fill = кратность относительно некоторого расстояния fill, кусками которого заполнены пустые места
\begin{flushright}
\ifdefined\supervisorTwoFio
Научные руководители: 


\supervisorRegalia

\ifdefined\supervisorDead
\framebox{\supervisorFio}
\else
\supervisorFio
\fi

\supervisorTwoRegalia

\ifdefined\supervisorTwoDead
\framebox{\supervisorTwoFio}
\else
\supervisorFio
\fi
\else
Научный руководитель:

\supervisorRegalia

\ifdefined\supervisorDead
\framebox{\supervisorFio}
\else
\supervisorFio
\fi
\fi

\end{flushright}
%
\vspace{0pt plus4fill} %число перед fill = кратность относительно некоторого расстояния fill, кусками которого заполнены пустые места
{\centering\thesisCity\ "--- \thesisYear\par}
           % Титульный лист
\include{Dissertation/contents}        % Оглавление
\ifnumequal{\value{contnumfig}}{1}{}{\counterwithout{figure}{chapter}}
\ifnumequal{\value{contnumtab}}{1}{}{\counterwithout{table}{chapter}}
\chapter*{Введение}                         % Заголовок
\addcontentsline{toc}{chapter}{Введение}    % Добавляем его в оглавление

Данная работа посвящена изучению параметров, характеризующих предварительное напряженное состояние грунта. Планируется изучить большое количество как современных работ, так и ранних исследований в этой сфере. Такие работы как Казагранде, Беккера, Терцаги и других, имеют огромное значение для современного грунтоведения и механики грунтов по сей день.
В настоящее время в нашей стране только начинается более подробное изучение данного вопроса, о чем говорит введение нового стандарта ГОСТ. 

Все больше в практику проектирования фундаментов сооружений входят расчеты грунтовых оснований с использованием численного моделирования поведения грунтов на базе метода конечных элементов. 
В последнее время они все чаще используются в практике проектирования промышленно-гражданских сооружений. 

Расчеты выполняются в таких программных комплексах, как PLAXIS, MIDAS и др. 
Для повышения достоверности расчетов данные программы наряду со стандартными характеристиками грунтов используют показатели, не отраженные в отечественных нормах. 

В частности, это параметры, характеризующие предварительное напряженное состояние грунта. 
К ним относятся в том числе такие показатели, как:
\begin{itemize}
    \item напряжение предуплотнения $\sigma_p$ (preconsolidation pressure);
    \item напряжение переуплотнения $POP$ (preoverburden pressure);
    \item коэффициент переуплотнения $OCR$ (overconsolidation ratio).
\end{itemize}

Напряжение переуплотнения $POP$ "--- разница между эффективным напряжением предвари­тельного уплотнения $\sigma_p$ и вертикальным эффективным напряжением от собственного веса грунта $\sigma_0$.
$$POP = \sigma_p - \sigma_0$$
Коэффициент переуплотнения $OCR$ "---  отношение эффективного напряжения предваритель­ного уплотнения $\sigma_p$ к вертикальному эффективному напряжению от собственного веса грунта $\sigma_0$.
$$OCR = \frac{\sigma_p}{\sigma_0}$$


Эти параметры нужны чтобы более детально описывать поведение грунтов, вычислительные мощности позволяют. 
Такие модели, как Cam Clay, Soft Soil, Soft Soil Creep, Hardening Soil и прочие, используют эти параметры. 

Переуплотненные грунты широко распространены как на материковой части, так и на шельфе северных морей, в частности на арктическом и дальневосточном шельфе России.
Осадочные грунты, такие как аргиллиты или алевролиты, выше которых отложения отсутствуют или имеют небольшую мощность, являются хорошими примерами переуплотненных грунтов.
Поэтому, при инженерно-геологических изысканиях особенно важно определять параметры грунта, связанные с его предварительным напряженным состоянием.

В настоящее время существует около полусотни  методов определения напряжения предуплотнения $\sigma_p$. (Крамаренко, 2014) \cite{kram2014}
Однако, разработчиками ГОСТ 58326-2018 было принято решение включить только два метода: Казагранде и Беккера, которые и будут рассмотрены в данной курсовой работе. 

Основной целью данной работы является сравнение точности определения $\sigma_p$ и, соответственно, $OSR$ методами Беккера и Казагранде, выявление их недостатков на примере испытаний, а также лабораторные определения состава грунта и характеристик основных физических и физико-механических свойств. 
Определение точности этих методов будет осуществляться на искусственных образцах глинистых грунтов четвертичного возраста с заданным напряжением переуплотнения. 
В работе будет рассматриваться единственный фактор максимального исторического напряжения. Прочие факторы, при которых грунт может получить переуплотненное состояние, минуя большие нагрузки в своем прошлом, такие как литогенез, превращения минералов, выщелачивание и прочие не будут учитываться.

\textbf{Цели.}
Для достижения данной цели будут проведены следующие испытания:
\begin{itemize}
    \item отбор проб и пробоподготовка; 
    \item определение твердой компоненты грунта;
    \item химический анализ образцов;
    \item гранулометрический анализ образцов;
    \item рентгено-структурный анализ образцов;
    \item подготовка к компрессионным испытаниям и их осуществление;
    \item анализ и обработка полученных данных методами Казагранде и Беккера;
    \item оптимизация и автоматизация процессов определения напряжения предуплотнения грунтов $\sigma_p$;
    \item изучение международных стандартов.
\end{itemize}

%ознакомление с наиболее известными зарубежными и отечественными методами определения параметров переуплотнения грунтов,

Образцы для исследований были отобраны в ходе инженерно-геологических изысканий под ряд крупных объектов, расположенных в юго-западном районе Москвы НАО, поселение Сосенское, вблизи д. Николо-Хованское, под строящийся жилой комплекс <<Саларьево-парк>> группы компаний <<ПИК>>. 
 

\textbf{Благодарности.}

Особую благодарность хочу выразить: моему научному руководителю В.\;В.\;Шаниной, а также главному специалисту лаборатории ООО <<ГеоГрадСтрой>> В.\;В.\;Матвееву за бесценную помощь и поддержку на всех этапах исследований; сотруднику лаборатории механики грунтов В.\;С.\;Чочиава-Степаненко за помощь в компрессионных испытаниях; В.\;В.\;Крупской и С.\;А.\;Гараниной за выполнение и обработку данных рентгено-структурного анализа.
Испытания планируется провести с октября 2019 г. по март 2020 г. Обработка будет выполняться по мере завершения испытаний.    % Введение
\ifnumequal{\value{contnumfig}}{1}{\counterwithout{figure}{chapter}
}{\counterwithin{figure}{chapter}}
\ifnumequal{\value{contnumtab}}{1}{\counterwithout{table}{chapter}
}{\counterwithin{table}{chapter}}
\chapter{Оформление различных элементов}\label{ch:ch1}

\section{Форматирование текста}\label{sec:ch1/sec1}

Мы можем сделать \textbf{жирный текст} и \textit{курсив}.

\section{Ссылки}\label{sec:ch1/sec2}

Сошлёмся на библиографию.
Одна ссылка: \cite[с.~54]{Sokolov}\cite[с.~36]{Gaidaenko}.
Две ссылки: \cite{Sokolov,Gaidaenko}.
Ссылка на собственные работы: \cite{vakbib1, confbib2}.
Много ссылок: %\cite[с.~54]{Lermontov,Management,Borozda} % такой «фокус»
%вызывает biblatex warning относительно опции sortcites, потому что неясно, к
%какому источнику относится уточнение о страницах, а bibtex об этой проблеме
%даже не предупреждает
\cite{Lermontov, Management, Borozda, Marketing, Constitution, FamilyCode,
Gost.7.0.53, Razumovski, Lagkueva, Pokrovski, Methodology, Nasirova, Berestova,
Kriger}%
\ifnumequal{\value{bibliosel}}{0}{% Примеры для bibtex8
    \cite{Sirotko, Lukina, Encyclopedia}%
}{% Примеры для biblatex через движок biber
    \cite{Sirotko2, Lukina2, Encyclopedia2}%
}%
.
И~ещё немного ссылок:~\cite{Article,Book,Booklet,Conference,Inbook,Incollection,Manual,Mastersthesis,
Misc,Phdthesis,Proceedings,Techreport,Unpublished}
% Следует обратить внимание, что пробел после запятой внутри \cite{}
% обрабатывается ожидаемо, а пробел перед запятой, может вызывать проблемы при
% обработке ссылок.
\cite{medvedev2006jelektronnye, CEAT:CEAT581, doi:10.1080/01932691.2010.513279,
Gosele1999161,Li2007StressAnalysis, Shoji199895, test:eisner-sample,
test:eisner-sample-shorted, AB_patent_Pomerantz_1968, iofis_patent1960}
\ifnumequal{\value{bibliosel}}{0}{% Примеры для bibtex8
}{% Примеры для biblatex через движок biber
    \cite{patent2h, patent3h, patent2}%
}%
.

\ifnumequal{\value{bibliosel}}{0}{% Примеры для bibtex8
Попытка реализовать несколько ссылок на конкретные страницы
для \texttt{bibtex} реализации библиографии:
[\citenum{Sokolov}, с.~54; \citenum{Gaidaenko}, с.~36].
}{% Примеры для biblatex через движок biber
Несколько источников (мультицитата):
% Тут специально написано по-разному тире, для демонстрации, что
% применение специальных тире в настоящий момент в biblatex приводит к непоказу
% "с.".
\cites[vii--x, 5, 7]{Sokolov}[v"--~x, 25, 526]{Gaidaenko}[vii--x, 5, 7]{Techreport},
работает только в \texttt{biblatex} реализации библиографии.
}%

Ссылки на собственные работы:~\cite{vakbib1, confbib1}

Сошлёмся на приложения: Приложение~\ref{app:A}, Приложение~\ref{app:B2}.

Сошлёмся на формулу: формула~\eqref{eq:equation1}.

Сошлёмся на изображение: рисунок~\ref{fig:knuth}.

Стандартной практикой является добавление к ссылкам префикса, характеризующего тип элемента.
Это не является строгим требованием, но~позволяет лучше ориентироваться в документах большого размера.
Например, для ссылок на~рисунки используется префикс \textit{fig},
для ссылки на~таблицу "--- \textit{tab}.

В таблице~\ref{tab:tab_pref} приложения~\ref{app:B4} приведён список рекомендуемых
к использованию стандартных префиксов.

\section{Формулы}\label{sec:ch1/sec3}

Благодаря пакету \textit{icomma}, \LaTeX~одинаково хорошо воспринимает
в~качестве десятичного разделителя и запятую (\(3,1415\)), и точку (\(3.1415\)).

\subsection{Ненумерованные одиночные формулы}\label{subsec:ch1/sec3/sub1}

Вот так может выглядеть формула, которую необходимо вставить в~строку
по~тексту: \(x \approx \sin x\) при \(x \to 0\).

А вот так выглядит ненумерованная отдельностоящая формула c подстрочными
и надстрочными индексами:
\[
(x_1+x_2)^2 = x_1^2 + 2 x_1 x_2 + x_2^2
\]

Формула с неопределенным интегралом:
\[
\int f(\alpha+x)=\sum\beta
\]

При использовании дробей формулы могут получаться очень высокие:
\[
  \frac{1}{\sqrt{2}+
  \displaystyle\frac{1}{\sqrt{2}+
  \displaystyle\frac{1}{\sqrt{2}+\cdots}}}
\]

В формулах можно использовать греческие буквы:
%Все \original... команды заранее, ради этого примера, определены в Dissertation\userstyles.tex
\[
\alpha\beta\gamma\delta\originalepsilon\epsilon\zeta\eta\theta%
\vartheta\iota\kappa\varkappa\lambda\mu\nu\xi\pi\varpi\rho\varrho%
\sigma\varsigma\tau\upsilon\originalphi\phi\chi\psi\omega\Gamma\Delta%
\Theta\Lambda\Xi\Pi\Sigma\Upsilon\Phi\Psi\Omega
\]
\[%https://texfaq.org/FAQ-boldgreek
\boldsymbol{\alpha\beta\gamma\delta\originalepsilon\epsilon\zeta\eta%
\theta\vartheta\iota\kappa\varkappa\lambda\mu\nu\xi\pi\varpi\rho%
\varrho\sigma\varsigma\tau\upsilon\originalphi\phi\chi\psi\omega\Gamma%
\Delta\Theta\Lambda\Xi\Pi\Sigma\Upsilon\Phi\Psi\Omega}
\]

Для добавления формул можно использовать пары \verb+$+\dots\verb+$+ и \verb+$$+\dots\verb+$$+,
но~они считаются устаревшими.
Лучше использовать их функциональные аналоги \verb+\(+\dots\verb+\)+ и \verb+\[+\dots\verb+\]+.

\subsection{Ненумерованные многострочные формулы}\label{subsec:ch1/sec3/sub2}

Вот так можно написать две формулы, не нумеруя их, чтобы знаки <<равно>> были
строго друг под другом:
\begin{align}
  f_W & =  \min \left( 1, \max \left( 0, \frac{W_{soil} / W_{max}}{W_{crit}} \right)  \right), \nonumber \\
  f_T & =  \min \left( 1, \max \left( 0, \frac{T_s / T_{melt}}{T_{crit}} \right)  \right), \nonumber
\end{align}

Выровнять систему ещё и по переменной \( x \) можно, используя окружение
\verb|alignedat| из пакета \verb|amsmath|. Вот так:
\[
    |x| = \left\{
    \begin{alignedat}{2}
        &&x, \quad &\text{eсли } x\geqslant 0 \\
        &-&x, \quad & \text{eсли } x<0
    \end{alignedat}
    \right.
\]
Здесь первый амперсанд (в исходном \LaTeX\ описании формулы) означает
выравнивание по~левому краю, второй "--- по~\( x \), а~третий "--- по~слову
<<если>>. Команда \verb|\quad| делает большой горизонтальный пробел.

Ещё вариант:
\[
    |x|=
    \begin{cases}
    \phantom{-}x, \text{если } x \geqslant 0 \\
    -x, \text{если } x<0
    \end{cases}
\]

Кроме того, для  нумерованных формул \verb|alignedat| делает вертикальное
выравнивание номера формулы по центру формулы. Например, выравнивание
компонент вектора:
\begin{equation}
\label{eq:2p3}
\begin{alignedat}{2}
{\mathbf{N}}_{o1n}^{(j)} = \,{\sin} \phi\,n\!\left(n+1\right)
         {\sin}\theta\,
         \pi_n\!\left({\cos} \theta\right)
         \frac{
               z_n^{(j)}\!\left( \rho \right)
              }{\rho}\,
           &{\boldsymbol{\hat{\mathrm e}}}_{r}\,+   \\
+\,
{\sin} \phi\,
         \tau_n\!\left({\cos} \theta\right)
         \frac{
            \left[\rho z_n^{(j)}\!\left( \rho \right)\right]^{\prime}
              }{\rho}\,
            &{\boldsymbol{\hat{\mathrm e}}}_{\theta}\,+   \\
+\,
{\cos} \phi\,
         \pi_n\!\left({\cos} \theta\right)
         \frac{
            \left[\rho z_n^{(j)}\!\left( \rho \right)\right]^{\prime}
              }{\rho}\,
            &{\boldsymbol{\hat{\mathrm e}}}_{\phi}\:.
\end{alignedat}
\end{equation}

Ещё об отступах. Иногда для лучшей <<читаемости>> формул полезно
немного исправить стандартные интервалы \LaTeX\ с учётом логической
структуры самой формулы. Например в формуле~\ref{eq:2p3} добавлен
небольшой отступ \verb+\,+ между основными сомножителями, ниже
результат применения всех вариантов отступа:
\begin{align*}
\backslash! &\quad f(x) = x^2\! +3x\! +2 \\
  \mbox{по-умолчанию} &\quad f(x) = x^2+3x+2 \\
\backslash, &\quad f(x) = x^2\, +3x\, +2 \\
\backslash{:} &\quad f(x) = x^2\: +3x\: +2 \\
\backslash; &\quad f(x) = x^2\; +3x\; +2 \\
\backslash \mbox{space} &\quad f(x) = x^2\ +3x\ +2 \\
\backslash \mbox{quad} &\quad f(x) = x^2\quad +3x\quad +2 \\
\backslash \mbox{qquad} &\quad f(x) = x^2\qquad +3x\qquad +2
\end{align*}

Можно использовать разные математические алфавиты:
\begin{align}
\mathcal{ABCDEFGHIJKLMNOPQRSTUVWXYZ} \nonumber \\
\mathfrak{ABCDEFGHIJKLMNOPQRSTUVWXYZ} \nonumber \\
\mathbb{ABCDEFGHIJKLMNOPQRSTUVWXYZ} \nonumber
\end{align}

Посмотрим на систему уравнений на примере аттрактора Лоренца:

\[
\left\{
  \begin{array}{rl}
    \dot x = & \sigma (y-x) \\
    \dot y = & x (r - z) - y \\
    \dot z = & xy - bz
  \end{array}
\right.
\]

А для вёрстки матриц удобно использовать многоточия:
\[
\left(
  \begin{array}{ccc}
    a_{11} & \ldots & a_{1n} \\
    \vdots & \ddots & \vdots \\
    a_{n1} & \ldots & a_{nn} \\
  \end{array}
\right)
\]

\subsection{Нумерованные формулы}\label{subsec:ch1/sec3/sub3}

А вот так пишется нумерованная формула:
\begin{equation}
  \label{eq:equation1}
  e = \lim_{n \to \infty} \left( 1+\frac{1}{n} \right) ^n
\end{equation}

Нумерованных формул может быть несколько:
\begin{equation}
  \label{eq:equation2}
  \lim_{n \to \infty} \sum_{k=1}^n \frac{1}{k^2} = \frac{\pi^2}{6}
\end{equation}

Впоследствии на формулы~\eqref{eq:equation1} и~\eqref{eq:equation2} можно ссылаться.

Сделать так, чтобы номер формулы стоял напротив средней строки, можно,
используя окружение \verb|multlined| (пакет \verb|mathtools|) вместо
\verb|multline| внутри окружения \verb|equation|. Вот так:
\begin{equation} % \tag{S} % tag - вписывает свой текст
  \label{eq:equation3}
    \begin{multlined}
        1+ 2+3+4+5+6+7+\dots + \\
        + 50+51+52+53+54+55+56+57 + \dots + \\
        + 96+97+98+99+100=5050
    \end{multlined}
\end{equation}

Используя команду \verb|\eqrefs|, можно
красиво ссылаться сразу на несколько формул
\eqrefs{eq:equation1, eq:equation3, eq:equation2}, даже перепутав
порядок ссылок \verb|\eqrefs{eq1, eq3, eq2}|.
Аналогично, для ссылок на несколько рисунков, таблиц и~т.\:д.
\refs{sec:ch1/sec1, sec:ch1/sec2, sec:ch1/sec3} можно использовать
команду \verb|\refs|.
Обе эти команды определены в файле \verb|common/packages.tex|.

Уравнения~\eqrefs{eq:subeq_1,eq:subeq_2} демонстрируют возможности
окружения \verb|\subequations|.
\begin{subequations}
    \label{eq:subeq_1}
    \begin{gather}
        y = x^2 + 1 \label{eq:subeq_1-1} \\
        y = 2 x^2 - x + 1 \label{eq:subeq_1-2}
    \end{gather}
\end{subequations}
Ссылки на отдельные уравнения~\eqrefs{eq:subeq_1-1,eq:subeq_1-2,eq:subeq_2-1}.
\begin{subequations}
    \label{eq:subeq_2}
    \begin{align}
        y &= x^3 + x^2 + x + 1 \label{eq:subeq_2-1} \\
        y &= x^2
    \end{align}
\end{subequations}

\subsection{Форматирование чисел и размерностей величин}\label{sec:units}

Числа форматируются при помощи команды \verb|\num|:
\num{5,3};
\num{2,3e8};
\num{12345,67890};
\num{2,6 d4};
\num{1+-2i};
\num{.3e45};
\num[exponent-base=2]{5 e64};
\num[exponent-base=2,exponent-to-prefix]{5 e64};
\num{1.654 x 2.34 x 3.430}
\num{1 2 x 3 / 4}.
Для написания последовательности чисел можно использовать команды \verb|\numlist| и \verb|\numrange|:
\numlist{10;30;50;70}; \numrange{10}{30}.
Значения углов можно форматировать при помощи команды \verb|\ang|:
\ang{2.67};
\ang{30,3};
\ang{-1;;};
\ang{;-2;};
\ang{;;-3};
\ang{300;10;1}.

Обратите внимание, что ГОСТ запрещает использование знака <<->> для обозначения отрицательных чисел
за исключением формул, таблиц и~рисунков.
Вместо него следует использовать слово <<минус>>.

Размерности можно записывать при помощи команд \verb|\si| и \verb|\SI|:
\si{\farad\squared\lumen\candela};
\si{\joule\per\mole\per\kelvin};
\si[per-mode = symbol-or-fraction]{\joule\per\mole\per\kelvin};
\si{\metre\per\second\squared};
\SI{0.10(5)}{\neper};
\SI{1.2-3i e5}{\joule\per\mole\per\kelvin};
\SIlist{1;2;3;4}{\tesla};
\SIrange{50}{100}{\volt}.
Список единиц измерений приведён в таблицах~\refs{tab:unit:base,
tab:unit:derived,tab:unit:accepted,tab:unit:physical,tab:unit:other}.
Приставки единиц приведены в~таблице~\ref{tab:unit:prefix}.

С дополнительными опциями форматирования можно ознакомиться в~описании пакета \texttt{siunitx};
изменить или добавить единицы измерений можно в~файле \texttt{siunitx.cfg}.

\begin{table}
    \centering
    \captionstyle{\centering}
    \caption{Основные величины СИ}\label{tab:unit:base}
    \begin{tabular}{llc}
        \toprule
        Название  & Команда                & Символ         \\
        \midrule
        Ампер     & \verb|\ampere| & \si{\ampere}   \\
        Кандела   & \verb|\candela| & \si{\candela}  \\
        Кельвин   & \verb|\kelvin| & \si{\kelvin}   \\
        Килограмм & \verb|\kilogram| & \si{\kilogram} \\
        Метр      & \verb|\metre| & \si{\metre}    \\
        Моль      & \verb|\mole| & \si{\mole}     \\
        Секунда   & \verb|\second| & \si{\second}   \\
        \bottomrule
    \end{tabular}
\end{table}

\begin{table}
  \small
  \centering
  \begin{threeparttable}% выравнивание подписи по границам таблицы
    \caption{Производные единицы СИ}\label{tab:unit:derived}
    \begin{tabular}{llc|llc}
        \toprule
        Название       & Команда                 & Символ              & Название & Команда & Символ \\
        \midrule
        Беккерель      & \verb|\becquerel|  & \si{\becquerel}     &
        Ньютон         & \verb|\newton|  & \si{\newton}                                      \\
        Градус Цельсия & \verb|\degreeCelsius| & \si{\degreeCelsius} &
        Ом             & \verb|\ohm| & \si{\ohm}                                         \\
        Кулон          & \verb|\coulomb| & \si{\coulomb}       &
        Паскаль        & \verb|\pascal| & \si{\pascal}                                      \\
        Фарад          & \verb|\farad| & \si{\farad}         &
        Радиан         & \verb|\radian| & \si{\radian}                                      \\
        Грей           & \verb|\gray| & \si{\gray}          &
        Сименс         & \verb|\siemens| & \si{\siemens}                                     \\
        Герц           & \verb|\hertz| & \si{\hertz}         &
        Зиверт         & \verb|\sievert| & \si{\sievert}                                     \\
        Генри          & \verb|\henry| & \si{\henry}         &
        Стерадиан      & \verb|\steradian| & \si{\steradian}                                   \\
        Джоуль         & \verb|\joule| & \si{\joule}         &
        Тесла          & \verb|\tesla| & \si{\tesla}                                       \\
        Катал          & \verb|\katal| & \si{\katal}         &
        Вольт          & \verb|\volt| & \si{\volt}                                        \\
        Люмен          & \verb|\lumen| & \si{\lumen}         &
        Ватт           & \verb|\watt| & \si{\watt}                                        \\
        Люкс           & \verb|\lux| & \si{\lux}           &
        Вебер          & \verb|\weber| & \si{\weber}                                       \\
        \bottomrule
    \end{tabular}
  \end{threeparttable}
\end{table}

\begin{table}
  \centering
  \begin{threeparttable}% выравнивание подписи по границам таблицы
    \caption{Внесистемные единицы}\label{tab:unit:accepted}

    \begin{tabular}{llc}
        \toprule
        Название        & Команда                 & Символ          \\
        \midrule
        День            & \verb|\day| & \si{\day}       \\
        Градус          & \verb|\degree| & \si{\degree}    \\
        Гектар          & \verb|\hectare| & \si{\hectare}   \\
        Час             & \verb|\hour| & \si{\hour}      \\
        Литр            & \verb|\litre| & \si{\litre}     \\
        Угловая минута  & \verb|\arcminute| & \si{\arcminute} \\
        Угловая секунда & \verb|\arcsecond| & \si{\arcsecond} \\ %
        Минута          & \verb|\minute| & \si{\minute}    \\
        Тонна           & \verb|\tonne| & \si{\tonne}     \\
        \bottomrule
    \end{tabular}
  \end{threeparttable}
\end{table}

\begin{table}
    \centering
    \captionstyle{\centering}
    \caption{Внесистемные единицы, получаемые из эксперимента}\label{tab:unit:physical}
    \begin{tabular}{llc}
        \toprule
        Название                & Команда                 & Символ                 \\
        \midrule
        Астрономическая единица & \verb|\astronomicalunit| & \si{\astronomicalunit} \\
        Атомная единица массы   & \verb|\atomicmassunit| & \si{\atomicmassunit}   \\
        Боровский радиус        & \verb|\bohr| & \si{\bohr}             \\
        Скорость света          & \verb|\clight| & \si{\clight}           \\
        Дальтон                 & \verb|\dalton| & \si{\dalton}           \\
        Масса электрона         & \verb|\electronmass| & \si{\electronmass}     \\
        Электрон Вольт          & \verb|\electronvolt| & \si{\electronvolt}     \\
        Элементарный заряд      & \verb|\elementarycharge| & \si{\elementarycharge} \\
        Энергия Хартри          & \verb|\hartree| & \si{\hartree}          \\
        Постоянная Планка       & \verb|\planckbar| & \si{\planckbar}        \\
        \bottomrule
    \end{tabular}
\end{table}

\begin{table}
  \centering
  \begin{threeparttable}% выравнивание подписи по границам таблицы
    \caption{Другие внесистемные единицы}\label{tab:unit:other}
    \begin{tabular}{llc}
        \toprule
        Название                  & Команда                 & Символ             \\
        \midrule
        Ангстрем                  & \verb|\angstrom| & \si{\angstrom}     \\
        Бар                       & \verb|\bar| & \si{\bar}          \\
        Барн                      & \verb|\barn| & \si{\barn}         \\
        Бел                       & \verb|\bel| & \si{\bel}          \\
        Децибел                   & \verb|\decibel| & \si{\decibel}      \\
        Узел                      & \verb|\knot| & \si{\knot}         \\
        Миллиметр ртутного столба & \verb|\mmHg| & \si{\mmHg}         \\
        Морская миля              & \verb|\nauticalmile| & \si{\nauticalmile} \\
        Непер                     & \verb|\neper| & \si{\neper}        \\
        \bottomrule
    \end{tabular}
  \end{threeparttable}
\end{table}

\begin{table}
  \small
  \centering
  \begin{threeparttable}% выравнивание подписи по границам таблицы
    \caption{Приставки СИ}\label{tab:unit:prefix}
    \begin{tabular}{llcc|llcc}
        \toprule
        Приставка & Команда                 & Символ      & Степень &
        Приставка & Команда                 & Символ      & Степень   \\
        \midrule
        Иокто     & \verb|\yocto| & \si{\yocto} & -24     &
        Дека      & \verb|\deca| & \si{\deca}  & 1         \\
        Зепто     & \verb|\zepto| & \si{\zepto} & -21     &
        Гекто     & \verb|\hecto| & \si{\hecto} & 2         \\
        Атто      & \verb|\atto| & \si{\atto}  & -18     &
        Кило      & \verb|\kilo| & \si{\kilo}  & 3         \\
        Фемто     & \verb|\femto| & \si{\femto} & -15     &
        Мега      & \verb|\mega| & \si{\mega}  & 6         \\
        Пико      & \verb|\pico| & \si{\pico}  & -12     &
        Гига      & \verb|\giga| & \si{\giga}  & 9         \\
        Нано      & \verb|\nano| & \si{\nano}  & -9      &
        Терра     & \verb|\tera| & \si{\tera}  & 12        \\
        Микро     & \verb|\micro| & \si{\micro} & -6      &
        Пета      & \verb|\peta| & \si{\peta}  & 15        \\
        Милли     & \verb|\milli| & \si{\milli} & -3      &
        Екса      & \verb|\exa| & \si{\exa}   & 18        \\
        Санти     & \verb|\centi| & \si{\centi} & -2      &
        Зетта     & \verb|\zetta| & \si{\zetta} & 21        \\
        Деци      & \verb|\deci| & \si{\deci}  & -1      &
        Иотта     & \verb|\yotta| & \si{\yotta} & 24        \\
        \bottomrule
    \end{tabular}
  \end{threeparttable}
\end{table}

\subsection{Заголовки с формулами: \texorpdfstring{\(a^2 + b^2 = c^2\)}{%
a\texttwosuperior\ + b\texttwosuperior\ = c\texttwosuperior},
\texorpdfstring{\(\left\vert\textrm{{Im}}\Sigma\left(
\protect\varepsilon\right)\right\vert\approx const\)}{|Im\sigma (ε)| ≈ const},
\texorpdfstring{\(\sigma_{xx}^{(1)}\)}{\sigma\_\{xx\}\textasciicircum\{(1)\}}
}\label{subsec:with_math}

Пакет \texttt{hyperref} берёт текст для закладок в pdf-файле из~аргументов
команд типа \verb|\section|, которые могут содержать математические формулы,
а~также изменения цвета текста или шрифта, которые не отображаются в~закладках.
Чтобы использование формул в заголовках не вызывало в~логе компиляции появление
предупреждений типа <<\texttt{Token not allowed in~a~PDF string
(Unicode):(hyperref) removing...}>>, следует использовать конструкцию
\verb|\texorpdfstring{}{}|, где в~первых фигурных скобках указывается
формула, а~во~вторых "--- запись формулы для закладок.

\section{Рецензирование текста}\label{sec:markup}

В шаблоне для диссертации и автореферата заданы команды рецензирования.
Они видны при компиляции шаблона в режиме черновика или при установке
соответствующей настройки (\verb+showmarkup+) в~файле \verb+common/setup.tex+.

Команда \verb+\todo+ отмечает текст красным цветом.
\todo{Например, так.}

Команда \verb+\note+ позволяет выбрать цвет текста.
\note{Чёрный, } \note[red]{красный, } \note[green]{зелёный, }
\note[blue]{синий.} \note[orange]{Обратите внимание на ширину и расстановку
формирующихся пробелов, в~результате приведённой записи (зависит также
от~применяемого компилятора).}

Окружение \verb+commentbox+ также позволяет выбрать цвет.

\begin{commentbox}[red]
        Красный текст.

        Несколько параграфов красного текста.
\end{commentbox}

\begin{commentbox}[blue]
        Синяя формула.

        \begin{equation}
                \alpha + \beta = \gamma
        \end{equation}
\end{commentbox}

\verb+commentbox+ позволяет закомментировать участок кода в~режиме чистовика.
Чтобы убрать кусок кода для всех режимов, можно использовать окружение
\verb+comment+.

\begin{comment}
        Этот текст всегда скрыт.
\end{comment}
           % Глава 1
\include{Dissertation/part2}           % Глава 2
\include{Dissertation/part3}           % Глава 3
\chapter*{Заключение}                       % Заголовок
\addcontentsline{toc}{chapter}{Заключение}  % Добавляем его в оглавление

%% Согласно ГОСТ Р 7.0.11-2011:
%% 5.3.3 В заключении диссертации излагают итоги выполненного исследования, рекомендации, перспективы дальнейшей разработки темы.
%% 9.2.3 В заключении автореферата диссертации излагают итоги данного исследования, рекомендации и перспективы дальнейшей разработки темы.
%% Поэтому имеет смысл сделать эту часть общей и загрузить из одного файла в автореферат и в диссертацию:

В процессе исследования были проведены многочисленные опыты по определению физических, физико-механических свойств, гранулометрического, минералогического, химического составов грунтов и их закономерностей. 

В ходе проводимого исследования, удалось оценить и сравнить точность определения основных параметров переуплотненных грунтов методами Казагранде и Беккера.

Были выявлены слабые стороны методов определения переуплотнения, а также предложены меры по утранению недостатков этих методов и автоматизации их определения.


Исходя из вышесказанного, можно заявить, что поднятая проблема будет и дальше подвергаться изучению в будущем, так как в процессе поиска материалов были обнаружены и изучены более полусотни способов определения напряжения предуплотнения, однако, рассмотрение этих методов выходило за рамки данной работы.

Их изучению планируется посвятить дипломную работу, а также планируется изучить другие разновидности грунтов и накопить больше материала для статистической обработки.
      % Заключение
\include{Dissertation/acronyms}        % Список сокращений и условных обозначений
\include{Dissertation/dictionary}      % Словарь терминов
\clearpage                                  % В том числе гарантирует, что список литературы в оглавлении будет с правильным номером страницы
%\hypersetup{ urlcolor=black }               % Ссылки делаем чёрными
%\providecommand*{\BibDash}{}                % В стилях ugost2008 отключаем использование тире как разделителя
\urlstyle{rm}                               % ссылки URL обычным шрифтом
\ifdefmacro{\microtypesetup}{\microtypesetup{protrusion=false}}{} % не рекомендуется применять пакет микротипографики к автоматически генерируемому списку литературы
\insertbibliofull                           % Подключаем Bib-базы: все статьи единым списком
% Режим с подсписками
%\insertbiblioexternal                      % Подключаем Bib-базы: статьи, не являющиеся статьями автора по теме диссертации
% Для вывода выберите и расскомментируйте одно из двух
%\insertbiblioauthor                        % Подключаем Bib-базы: работы автора единым списком 
%\insertbiblioauthorgrouped                 % Подключаем Bib-базы: работы автора сгруппированные (ВАК, WoS, Scopus и т.д.)
\ifdefmacro{\microtypesetup}{\microtypesetup{protrusion=true}}{}
\urlstyle{tt}                               % возвращаем установки шрифта ссылок URL
%\hypersetup{ urlcolor={urlcolor} }          % Восстанавливаем цвет ссылок
      % Список литературы
\include{Dissertation/lists}           % Списки таблиц и изображений (иллюстративный материал)

%%% Настройки для приложений
\appendix
% Оформление заголовков приложений ближе к ГОСТ:
\setlength{\midchapskip}{20pt}
\renewcommand*{\afterchapternum}{\par\nobreak\vskip \midchapskip}
\renewcommand\thechapter{\Asbuk{chapter}} % Чтобы приложения русскими буквами нумеровались

\chapter{Таблица физико-механических свойст и гранулометрического состава}\label{app:tp}

\newpage

\newpage

\chapter{Графики компрессионных испытаний}\label{app:ct}
%% \small
% \pgfplotsset{
%     discard if not/.style 2 args={
%         x filter/.code={
%             \edef\tempa{\thisrow{#1}}
%             \edef\tempb{#2}
%             \ifx\tempa\tempb
%             \else
%                 \def\pgfmathresult{inf}
%             \fi
%         }
%     }
% }

\makeatletter
\pgfplotstableset{
	columns={Sigma,Epsilon},
    discard if not/.style 2 args={
        row predicate/.code={
            \def\pgfplotstable@loc@TMPd{\pgfplotstablegetelem{##1}{#1}\of}
            \expandafter\pgfplotstable@loc@TMPd\pgfplotstablename
            \edef\tempa{\pgfplotsretval}
            \edef\tempb{#2}
            \ifx\tempa\tempb
            \else
                \pgfplotstableuserowfalse
            \fi
        }
	},
	% Format numbers and titles
	columns/Sigma/.style = {column name = $\sigma_1$, fixed ,fixed zerofill, precision=0, set thousands separator={\,}, dec sep align},
	columns/Epsilon/.style = {column name = $\epsilon_1$, fixed, fixed zerofill, precision=3, dec sep align},
	%columns/p/.style    = {column name = $p$, fixed, fixed zerofill, dec sep align},
	%columns/ci/.style   = {string type, column name = 95\% CI},
}
\makeatother

% \pgfplotsset{
% 	%samples=15,
% 	width= 12cm, %\linewidth,
% 	height = 12cm,
% 	xlabel={Вертикальное эфф. напряжение $\sigma_1$, кПа},
% 	ylabel={Отн. верт. деформация $\epsilon_1$, д. е.},
% 	%extra y ticks={45},
% 	legend pos=north west,
% 	y dir=reverse, 
% 	y tick label style={
% 		/pgf/number format/.cd,
% 			fixed,
% 			fixed zerofill,
% 			precision=3,
% 		/tikz/.cd
% 	},
% 	x tick label style={
%         /pgf/number format/set thousands separator={\,},
% 		/pgf/number format/.cd,
% 			fixed,
% 			fixed zerofill
%         /tikz/.cd
% 	},
%     xticklabel={
%         \pgfkeys{ /pgf/number format/fixed, 
%             /pgf/number format/fixed zerofill, 
%             /pgf/number format/precision=0} 
%         \pgfmathparse{exp(\tick)}
%         \pgfmathprintnumber{\pgfmathresult}
%     }
% }

\pgfplotstableset{
begin table=\begin{longtable},
end table=\end{longtable},
col sep = semicolon,
header=true,    
columns={Sigma,Epsilon},
columns/Sigma/.style={fixed,fixed zerofill,precision=0,column type=r},
columns/Epsilon/.style={fixed,fixed zerofill,precision=3,column type=r},
every head row/.style={
before row=\caption{\texttt{GJ6835}}\\
\toprule,
after row=\midrule},
every last row/.style={
after row=\bottomrule},
}

\begin{multicols}{3}
% {
%\tiny
\pgfplotstabletypeset{data/GJ6835.csv}

\pgfplotstabletypeset{data/GJ6874.csv}

\pgfplotstabletypeset{data/GJ6890.csv}

\pgfplotstabletypeset{data/GJ6864.csv}

\pgfplotstabletypeset{data/GJ6838.csv}

\pgfplotstabletypeset{data/GJ6898.csv}

\pgfplotstabletypeset{data/GJ6888.csv}

\pgfplotstabletypeset{data/GJ68A0.csv}

\pgfplotstabletypeset{data/GJ6840.csv}

\pgfplotstabletypeset{data/GJ6895.csv}

\pgfplotstabletypeset{data/GJ6885.csv}

\pgfplotstabletypeset{data/GJ68B3.csv}

\end{multicols}
\small
\pgfplotsset{
    % discard if not/.style 2 args={
    %     x filter/.code={
    %         \edef\tempa{\thisrow{#1}}
    %         \edef\tempb{#2}
    %         \ifx\tempa\tempb
    %         \else
    %             \def\pgfmathresult{inf}
    %         \fi
    %     }
    % }
}

\pgfplotsset{
% 	%samples=15,
	width= \linewidth,
	height = 12cm,
	xlabel={Вертикальное эфф. напряжение $\sigma_1$, кПа},
	ylabel={Отн. верт. деформация $\epsilon_1$, д. е.},
% 	%extra y ticks={45},
	legend pos=north west,
	y dir=reverse, 
	y tick label style={
		/pgf/number format/.cd,
			fixed,
			fixed zerofill,
			precision=3,
		/tikz/.cd
	},
	ylabel near ticks,
	% ylabel shift = 5cm,
% 	% clip mode=individual,
	x tick label style={
        /pgf/number format/set thousands separator={\,},
		/pgf/number format/.cd,
			fixed,
			fixed zerofill
        /tikz/.cd
	},
    xticklabel={
        \pgfkeys{ /pgf/number format/fixed, 
            /pgf/number format/fixed zerofill, 
            /pgf/number format/precision=0} 
        \pgfmathparse{exp(\tick)}
        \pgfmathprintnumber{\pgfmathresult}
    }
}

{
\tiny

\begin{figure}
	\begin{tikzpicture}
		% \centering
		\begin{semilogxaxis}
		\addplot[mark=*, red] table [x=Sigma, y=Epsilon, col sep=semicolon] {data/GJ6835.csv};
		\end{semilogxaxis}
	\end{tikzpicture}
	\caption{Компрессионная кривая образца \texttt{GJ6835}}
\end{figure}

\begin{figure}
\begin{tikzpicture}
	\begin{semilogxaxis}[y dir=reverse]
	\addplot[mark=*, red] table [x=Sigma, y=Epsilon, col sep=semicolon] {data/GJ6874.csv};
	\end{semilogxaxis}
\end{tikzpicture}
\caption{\texttt{GJ6874}}
\end{figure}

\begin{figure}
\begin{tikzpicture}
	\begin{semilogxaxis}[y dir=reverse]
	\addplot[mark=*, red] table [x=Sigma, y=Epsilon, col sep=semicolon] {data/GJ6890.csv};
	\end{semilogxaxis}
\end{tikzpicture}
\caption{Компрессионная кривая образца \texttt{GJ6890}}
\end{figure}

\begin{figure}
\begin{tikzpicture}
	\begin{semilogxaxis}[y dir=reverse]
	\addplot[mark=*, red] table [x=Sigma, y=Epsilon, col sep=semicolon] {data/GJ6864.csv};
	\end{semilogxaxis}
\end{tikzpicture}
\caption{Компрессионная кривая образца \texttt{GJ6864}}
\end{figure}

\begin{figure}

\begin{tikzpicture}
	\begin{semilogxaxis}[y dir=reverse]
	\addplot[mark=*, red] table [x=Sigma, y=Epsilon, col sep=semicolon] {data/GJ6838.csv};
	\end{semilogxaxis}
\end{tikzpicture}
\caption{Компрессионная кривая образца \texttt{GJ6838}}
\end{figure}

\begin{figure}
\begin{tikzpicture}
	\begin{semilogxaxis}[y dir=reverse]
	\addplot[mark=*, red] table [x=Sigma, y=Epsilon, col sep=semicolon] {data/GJ6898.csv};
	\end{semilogxaxis}
\end{tikzpicture}
\caption{Компрессионная кривая образца \texttt{GJ6898}}
\end{figure}

\begin{figure}
\begin{tikzpicture}
	\begin{semilogxaxis}[y dir=reverse]
	\addplot[mark=*, red] table [x=Sigma, y=Epsilon, col sep=semicolon] {data/GJ6888.csv};
	\end{semilogxaxis}
\end{tikzpicture}
\caption{\texttt{GJ6888}}
\end{figure}

\begin{figure}
\begin{tikzpicture}
	\begin{semilogxaxis}[y dir=reverse]
	\addplot[mark=*, red] table [x=Sigma, y=Epsilon, col sep=semicolon] {data/GJ68A0.csv};
	\end{semilogxaxis}
\end{tikzpicture}
\caption{Компрессионная кривая образца \texttt{GJ68A0}}
\end{figure}

\begin{figure}
\begin{tikzpicture}
	\begin{semilogxaxis}[y dir=reverse]
	\addplot[mark=*, red] table [x=Sigma, y=Epsilon, col sep=semicolon] {data/GJ6840.csv};
	\end{semilogxaxis}
\end{tikzpicture}
\caption{Компрессионная кривая образца \texttt{GJ6840}}
\end{figure}

\begin{figure}
\begin{tikzpicture}
	\begin{semilogxaxis}[y dir=reverse]
	\addplot[mark=*, red] table [x=Sigma, y=Epsilon, col sep=semicolon] {data/GJ6895.csv};
	\end{semilogxaxis}§
\end{tikzpicture}
\caption{Компрессионная кривая образца \texttt{GJ6895}}
\end{figure}

\begin{figure}
\begin{tikzpicture}
	\begin{semilogxaxis}[y dir=reverse]
	\addplot[mark=*, red] table [x=Sigma, y=Epsilon, col sep=semicolon] {data/GJ6885.csv};
	\end{semilogxaxis}
\end{tikzpicture}
\caption{Компрессионная кривая образца \texttt{GJ6885}}
\end{figure}

\begin{figure}
\begin{tikzpicture}
	\begin{semilogxaxis}[y dir=reverse]
	\addplot[mark=*, red] table [x=Sigma, y=Epsilon, col sep=semicolon] {data/GJ68B3.csv};
	\end{semilogxaxis}
\end{tikzpicture}
\caption{Компрессионная кривая образца \texttt{GJ68B3}}
\end{figure}

}

\chapter{Обработка результатов}\label{app:methods}
% Inkscape figure
\begin{figure}[ht]
    {\centering
      %\def\svgwidth{5cm} % используем для изменения размера, если надо
      %\includesvg{figs/drawing}
      \small
      \subbottom[Метод Казагранде]{%
      \centering
      \input{images/oedometerCazagrande-1.pdf_tex} }
      \hfill 
      \\
      \hfill  
      \subbottom[Метод Беккера]{%
      \centering
      \input{images/oedometerBecker+1.pdf_tex}}
      \hfill 
      }
      \caption{Определение напряжения предуплотнения образца \texttt{GJ6835}}
      \label{img:6835}
    \end{figure}
    
    \begin{figure}
        {\centering
        \small
          %\def\svgwidth{5cm} % используем для изменения размера, если надо
          %\includesvg{figs/drawing}
          \subbottom[Метод Казагранде]{%
          \centering
          \input{images/oedometerCazagrande-2.pdf_tex} }
          \hfill 
          \\
          \hfill  
          \subbottom[Метод Беккера]{%
          \centering
          \input{images/oedometerBecker+2.pdf_tex}}
          \hfill 
          }
          \caption{Определение напряжения предуплотнения образца \texttt{GJ6874}}
          \label{img:6874}
    \end{figure}
    
    \begin{figure}
        {\centering
        \small
            %\def\svgwidth{5cm} % используем для изменения размера, если надо
            %\includesvg{figs/drawing}
            \subbottom[Метод Казагранде]{%
            \centering
            \input{images/oedometerCazagrande-3.pdf_tex} }
            \hfill 
            \\
            \hfill  
            \subbottom[Метод Беккера]{%
            \centering
            \input{images/oedometerBecker+3.pdf_tex}}
            \hfill 
            }
            \caption{Определение напряжения предуплотнения образца \texttt{GJ6890}}
            \label{img:6890}
    \end{figure}
    
    
    \begin{figure}
        {\centering
        \small
            %\def\svgwidth{5cm} % используем для изменения размера, если надо
            %\includesvg{figs/drawing}
            \subbottom[Метод Казагранде]{%
            \centering
            \input{images/oedometerCazagrande-4.pdf_tex} }
            \hfill 
            \\
            \hfill  
            \subbottom[Метод Беккера]{%
            \centering
            \input{images/oedometerBecker+4.pdf_tex}}
            \hfill 
            }
            \caption{Определение напряжения предуплотнения образца \texttt{GJ6864}}
            \label{img:6864}
    \end{figure}
    
    \begin{figure}
        {\centering
        \small
            %\def\svgwidth{5cm} % используем для изменения размера, если надо
            %\includesvg{figs/drawing}
            \subbottom[Метод Казагранде]{%
            \centering
            \input{images/oedometerCazagrande-5.pdf_tex} }
            \hfill 
            \\
            \hfill  
            \subbottom[Метод Беккера]{%
            \centering
            \input{images/oedometerBecker+5.pdf_tex}}
            \hfill 
            }
            \caption{Определение напряжения предуплотнения образца \texttt{GJ6838}}
            \label{img:6838}
    \end{figure}
    
    \begin{figure}
        {\centering
        \small
            %\def\svgwidth{5cm} % используем для изменения размера, если надо
            %\includesvg{figs/drawing}
            \subbottom[Метод Казагранде]{%
            \centering
            \input{images/oedometerCazagrande-6.pdf_tex} }
            \hfill 
            \\
            \hfill  
            \subbottom[Метод Беккера]{%
            \centering
            \input{images/oedometerBecker+6.pdf_tex}}
            \hfill 
            }
            \caption{Определение напряжения предуплотнения образца \texttt{GJ6898}}
            \label{img:6898}
    \end{figure}
    
    \begin{figure}
        {\centering
        \small
            %\def\svgwidth{5cm} % используем для изменения размера, если надо
            %\includesvg{figs/drawing}
            \subbottom[Метод Казагранде]{%
            \centering
            \input{images/oedometerCazagrande-7.pdf_tex} }
            \hfill 
            \\
            \hfill  
            \subbottom[Метод Беккера]{%
            \centering
            \input{images/oedometerBecker+7.pdf_tex}}
            \hfill 
            }
            \caption{Определение напряжения предуплотнения образца \texttt{GJ6888}}
            \label{img:6888}
    \end{figure}
    
    \begin{figure}
        {\centering
        \small
            %\def\svgwidth{5cm} % используем для изменения размера, если надо
            %\includesvg{figs/drawing}
            \subbottom[Метод Казагранде]{%
            \centering
            \input{images/oedometerCazagrande-8.pdf_tex} }
            \hfill 
            \\
            \hfill  
            \subbottom[Метод Беккера]{%
            \centering
            \input{images/oedometerBecker+8.pdf_tex}}
            \hfill 
            }
            \caption{Определение напряжения предуплотнения образца \texttt{GJ68A0}}
            \label{img:68A0}
    \end{figure}
    
    \begin{figure}
        {\centering
        \small
            %\def\svgwidth{5cm} % используем для изменения размера, если надо
            %\includesvg{figs/drawing}
            \subbottom[Метод Казагранде]{%
            \centering
            \input{images/oedometerCazagrande-9.pdf_tex} }
            \hfill 
            \\
            \hfill  
            \subbottom[Метод Беккера]{%
            \centering
            \input{images/oedometerBecker+9.pdf_tex}}
            \hfill 
        }
        \caption{Определение напряжения предуплотнения образца \texttt{GJ6840}}
        \label{img:6840}
    \end{figure}
    
    \begin{figure}
        {\centering
        \small
            %\def\svgwidth{5cm} % используем для изменения размера, если надо
            %\includesvg{figs/drawing}
            \subbottom[Метод Казагранде]{%
            \centering
            \input{images/oedometerCazagrande-10.pdf_tex} }
            \hfill 
            \\
            \hfill  
            \subbottom[Метод Беккера]{%
            \centering
            \input{images/oedometerBecker+10.pdf_tex}}
            \hfill  
            }
            \caption{Определение напряжения предуплотнения образца \texttt{GJ6895}}
            \label{img:6895}
    \end{figure}
    
    \begin{figure}
        {\centering
            %\def\svgwidth{5cm} % используем для изменения размера, если надо
            %\includesvg{figs/drawing}
            \small
            \subbottom[Метод Казагранде]{%
            \centering
            \input{images/oedometerCazagrande-11.pdf_tex} }
            \hfill 
            \\
            \hfill  
            \subbottom[Метод Беккера]{%
            \centering
            \input{images/oedometerBecker+11.pdf_tex}}
            \hfill  
            }
            \caption{Определение напряжения предуплотнения образца \texttt{GJ6885}}
            \label{img:6885}
    \end{figure}
    
    \begin{figure}
        {\centering
            %\def\svgwidth{5cm} % используем для изменения размера, если надо
            %\includesvg{figs/drawing}
            \small
            %\hfill
            \subbottom[Метод Казагранде]{%
            \centering
            \input{images/oedometerCazagrande-12.pdf_tex} }
            \hfill 
            \\
            \hfill   
            \subbottom[Метод Беккера]{%
            \centering
            \input{images/oedometerBecker+12.pdf_tex}}
            \hfill 
            }
            \caption{Определение напряжения предуплотнения образца \texttt{GJ68B3}}
            \label{img:68B3}
    \end{figure}        % Приложения

\end{document}
